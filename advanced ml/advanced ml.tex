\documentclass[11pt, a4page]{article}

\usepackage[utf8]{inputenc}
\usepackage{graphicx}
\usepackage[italian]{babel}
\usepackage{amsmath}
\usepackage{amssymb}
\usepackage{hyperref}
\usepackage{listings}
\usepackage[margin=1.2in]{geometry}
\usepackage{enumitem}


\title{\textbf{Advanced Machine Learning}}
\author{}
\date{}

\begin{document}
\maketitle
\begin{abstract}
  Il corso si concentra sull'uso di reti neurali (\textit{deep learning}).
  L'esame consiste in un progetto finale di gruppo più la valutazione degli \textit{assignment} svolti singolarmente durante l'anno per i frequentanti, altrimenti in un esame orale.
  Del progetto sarà valutata la relazione finale, in cui saranno specificati i compiti svolti dai singoli partecipanti, secondo delle specifiche definite.
\end{abstract}
\tableofcontents
\newpage

\section{Introduzione}
L'operazione di \textit{learning} si effettua per ottimizzare le performance di un algoritmo basandosi su un criterio prestabilito e una serie di esempi passati (esperienza) tramite \textit{reverse engeniering}.
Se l'operazione è definibile con un insieme di regole e la complessità non è particolarmente alta, non è richiesto l'uso del \textit{machine learning}; mentre quando le soluzioni cambiano (e il sistema si deve adattare) o la complessità è particolarmente alta, è necessario l'uso del \textit{machine learning}.

Di ogni algoritmo di \textit{machine learning} è importante la rappresentazione, la valutazione e l'ottimizzazione.
La scelta del criterio di valutazione è fondamentale perché impone la direzione in cui procedere nella fase di ottimizzazione per effettuare il processo di \textit{learning}; mentre l'algoritmo di ottimizzazione è scelto in base al compito che l'algoritmo deve svolgere. \newline

Il \textit{deep learning} deve la sua importanza alla facilità con cui sono stimate funzioni non lineari: per ottenere prestazioni simili in passato si modificava lo spazio delle \textit{features} applicando trasformazioni quali \textit{kernel} o PCA.
Questa sua caratteristica lo rende particolarmente adatto nell'analisi delle immagini, in cui sono usate più pezzi di immagine combinati tra di loro per effettuare l'analisi.
Il problema delle reti neurali fino al primo 2000 era la complessità dell'algoritmo di \textit{learning} (\textit{back propagation}), soppiantato da una versione più efficiente.

Suoi svantaggi d'altro canto sono la quantità di dati richiesta dal processo di apprendimento e dal costo dell'hardware richiesto (GPU e TPU).

L'uso del \textit{deep learning} è usato inoltre nell'analisi del linguaggio naturale, data l'evoluzione nel tempo di questo e dalle sue sfumature.
Altro successo è il \textit{reinfocement deep learning}, che suggerisce al computer come comportarsi in determinate circostanze in base all'evoluzione dell'ambiente e alle scelte pregresse.
\end{document}