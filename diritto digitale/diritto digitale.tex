\documentclass[a4page, 11pt, twocolumn]{article}

\usepackage{amsmath}
\usepackage[margin=0.7in,
            bottom=1.2in]{geometry}
\usepackage{eurosym}
%metto lui per gli accenti che non vedo nel pdf
\usepackage[utf8]{inputenc}
\usepackage{babel}

\title{Juridical Issues in Computer Science}
\date{}

\begin{document}
\maketitle

\section{Norme.}
Il diritto non può essere normato secondo schemi logici perché i consociati hanno il diritto di conoscere le norme: è necessario un linguaggio parlato e non formale.
Un linguaggio parlato è \textit{polisemico}, \textit{ambiguo} e \textit{vago}, per cui il legislatore tenta di aumentare la chiarezza attribuendo nuovi significati a parole già esistenti; mentre l'uso di standard scritti evita la perdita di informazione (rumore) nella trasmissione. \newline
Si definiscono più strati di norme:
\begin{itemize}
\item Regole tecniche: Sono lo strato più basso delle regole, sono alla pari delle leggi fisiche, non possono essere mutate e non dipendono dalla legislazione vigente. 
\item Standard (o \textit{soft-law}): Sono basate sulle regole tecniche e la comunità suggerisce il modo considerato migliore per una certa pratica(regole) (standard ISO per l'Europa).
\item Norme sociali: sono comportamenti diffusi in una data cultura; sono difficili da modificare anche quando vanno in contrasto con norme positive, riguardano l'uso che facciamo delle regole tecniche.
\item Norme positive: Sono lo strato più alto. La parola postive deriva dal latino e significa poste, sono le regole poste dal legislatore; possono anche essere in contrasto con le regole tecniche.
\end{itemize}
Il diritto è l'interazione tra i vari livelli di regole. \newline
È proprio a causa delle norme sociali che il legislatore non legifera su tecnologie \textit{distruptive}: non essendo noto l'uso, non sono poste regole nell'uso.

Si preferisce utilizzare regolamenti anzichè norme, per le questioni tecniche, per facilitare eventuali modifiche e rendere indipendente la questione da questioni politiche, poiché i regolamenti non devono essere approvati dal parlamento.

\section{Obsolescenza.}
L'obsolescenza digitale può avvenire per obsolescenza del supporto (dovuta a guasti meccanici o al tempo) o per obsolescenza software (il software utilizzato non è più supportato).
Software \textit{open source} e l'uso di formati liberi possono aumentare la longevità dell'informazione.

\section{Licenze free.}
Le licenze \textit{free} specificano solamente cosa non è permesso fare col software: qualsiasi altra azione è lecita.
Il termine \textit{free} non si riferisce appunto al prezzo ma al concetto di libertà: l'utente ha accesso a ogni parte del prodotto, è libero di modificarlo e di redistribuire il prodotto modificato (non però sotto licenze più restrittive).
Esistono vari tipi di licenze \textit{free} ma non sono per forza compatibili tra di loro.

Il copyright si differenzia dal diritto d'autore: il primo protegge l'editore garantendo protezione per gli investimenti effettuati, il secondo protegge l'autore garantendogli il controllo dell'opera anche dopo la pubblicazione.
Il diritto d'autore moderno si differenzia in diritti morali (inalienabili), che corrispondono all'idea originale del  diritto d'autore, e in diritti materiali, che normano come l'opera è accessibile al pubblico (o come sono gestite le opere derivate) che si sovrapone con l'idea del Copyright.

\subsection{Storia del \textit{copyright}.}
Prima della stampa (prima metà del XVI secolo), un testo scritto era considerato di pubblico dominio; ma con le prime pubblicazioni il sapere si trasmette pressochè invariato nel tempo (senza il \textit{rumore} della forma orale).
Il concetto di cultura è cambiato con l'invenzione del libro: indipendentemente dal messaggio contenuto, il sapere è molto più diffuso rendendo difficile dalle autorità controllare le opere prodotte ed è possibile fare più copie di un libro evitando spostamenti per consultarli.
Un autore diffondeva il proprio manoscritto a più stampatori per permettere una maggior diffusione dell'opera; ma nel 1709 l'Inghilterra emana il primo editto sul \textit{copyright} garantendo allo stampatore l'esclusiva: dato l'elevato costo delle apparecchiature, era così riconosciuto all'editore il diritto di proteggere il proprio investimento.

Successivamente, il diritto d'autore nasce con Kant: dopo la pubblicazione di \textit{Critica della Ragion Pura}, circolavano copie modificate da terzi spacciate per originali.
Il diritto d'autore è inalienabile e garantisce la paternità dell'opera mentre i diritti materiali normano lo sfruttamento dell'opera. \newline
Precedentemente, gli artisti vivevano sotto la protezione di un Mecenate; ma grazie a questa istituzione un artista poteva vivere delle sue opere, rispondendo univamente al pubblico: i primi furono Beethoven e Manzoni.
Il diritto d'autore era stato pensato per durare vita natural durante dell'autore e per i successivi 50 anni (70 nel caso di persone giuridiche) ma, negli anni '80 è stato aumentato di 20 anni (25 nel caso di persone giuridiche) a causa di pressioni della Walt Disney per proteggere il prodotto di Topolino.

Col tempo l'intermediario acquista potere (la distribuzione costa più della metà delle spese complessive), ma l'avvento di Internet modifica leggermente le cose: da una parte i \textit{blog} risparmiano agli autori l'uso del linguaggio HTML per pubblicare, dall'altra nascono nuovi intermediari digitali, non regolamentati.
Umberto Eco afferma che con l'avvento di Internet il sapere si appiattisce: è trasmesso tutto allo stesso modo, mentre prima erano selezionati solamente i testi considerati degni di nota dai sapienti del periodo.

\subsection{Storia delle licenze software.}
Negli anni '60, IBM iniziava a vendere i suoi sistemi UNIX indipendenti dall'hardware: è il primo caso di commercializzazione del software.
Tuttavia nell'ambito di ricerca il software è considerato bene comune e viene scambiato gratuitamente senza conseguenze legali: non si applica l'idea di furto perchè non è rimossa la copia originale ma solamente copiata.

Nel 1978, Bill Gates chiede che gli sviluppatori fossero tutelati ricevendo pagamenti a seguito della vendita del software prodotto. Nello stesso anno, negli USA si norma la diffusione del software che segue la medesima fattispecie della musica (ormai in musicassette).
Sempre in quegli anni, Bill Gates firma un contratto con IBM per l'installazione di DOS sui PC; successivamente IBM, per rimediare a tale errore, tenta il lancio di OS2, ma senza successo.

Richard Stallman invece teorizza l'informazione libera e le licenze free: chi ha le competenze può modificare un codice sorgente e ridistribuirlo.
Nel 1985 fonda la \textit{Free Software Foundation} scrivendo la prima licenza \textit{free} (la GPL[\textbf{G}eneral \textbf{P}ublic \textbf{L}icense]): è esplicitato solamente cosa è vietato. La sua battaglia è una battaglia politica ma mancava un pezzo fondamentale: Il Sistema Operativo che porta avanti il progetto è però sviluppato da Linus Torvalds: egli voleva progettare il miglior sistema operativo esistente, e già le prime versioni di Linux ebbero successo, se pur in una nicchia di esperti, per la stabilità che offriva. \newline

Nel 2002 Edgar David Villanueva Nunez, parlamentare peruviano, propone l'uso di software libero (a parità di qualità) nella digitalizzazione della Pubblica Amministrazione.
Microsoft Perù risponde affermando che il costo delle licenze è irrisorio e cita il caso del Messico che aveva informatizzato le scuole messicane con dei computer Linux (quando nessuno aveva le competenze per usarli).
Ma l'uso del software libero, afferma Villanueva, è per questioni politiche: è l'unico modo per garantire accesso libero, persistenza dei dati e sicurezza. \newline

La digitalizzazione della Pubblica Amministrazione italiana subisce pressioni da Microsoft Italia che offre licenze gratuite.

\subsection{Creative Commons.}
L'idea di licenza \textit{open} è stata portata a inizio 2000 al mondo analogico grazie alla licenze \textit{Creative Commons}(CC), che nascono da un idea di Lessig Lawrence un professore di Stanford, pensata per opere artistiche: il concetto di \textit{copyright} è uscito dalla sfera di competenza per cui è stato pensato, rendendo difficile se non impossibile la creazione artistica.
Il termine \textit{Commons} fa riferimento ai beni che, nel diritto anglosassone, non possono essere ridotti a proprietà privata perchè sono di pubblica utilità.
Ma non diventano neanche un public domain infatti io posso recintare le mie opere con diverse modalità di licenze:
\begin{enumerate}
\item  \textbf{Solo attribuzione} : Le CC non sono delle public domain e inoltre ha come condizione almeno quella di attribuire l’opera all’autore principale nella sua integrità (nella forma originale). In questo caso uno può fare quello che vuole con la mia opera basta che mi attribuisca l’opera.
\item \textbf{Attribuzione} opere\textbf{ non derivate} : Posso scegliere che non possano esistere delle opere derivate dalle mie senza il mio permesso, in tal caso uno come Mondadori può stampare la mia opera nella sua integrità, attribuendola a me e non mi deve dare i soldi (può convenire se sono un artista alle prime armi o se sono uno scienziato o un accademico come Kant).
\item \textbf{Attribuzione} opere \textbf{non derivate} \textbf{non commerciali} : Posso aggiungere anche la non commerciabilità, in questo caso tu puoi pubblicarla ma non puoi far pagare il libro senza il mio consenso (possiamo decidere insieme il prezzo ma non lui da solo).
\item \textbf{Attribuzione non commerciale} : Permetto tutte le modifiche e derivazioni affinché tu non la vendi. Se la vendi devi prendere il mio permesso.
\item \textbf{Attribuzione non commericale condividi allo stesso modo} : Permetto la modifica e la sua divulgazione ma devi ridistribuirla con la stessa licenza con la quale l’ho dato a te e per modificare la licenza devi prendere permesso da me.
\item \textbf{Attribuzione non commericale condividi allo stesso modo} : Permetto la modifica e la sua divulgazione ma devi ridistribuirla con la stessa licenza con la quale l’ho dato a te e per modificare la licenza devi prendere permesso da me.
\item \textbf{Attribuzione condividi allo stesso modo} : Semplicemente non posso chiudere l’opera che sto mandando in giro, non posso recintarla in nessun modo, l’unica cosa che è che deve essere distribuito allo stesso modo in cui te l’ho diffusa io. In senso tutti gli altri possono prendere l’opera derivata e venderla se l’ho permesso io nell’opera originaria.
\end{enumerate}



\subsection{Open Access.}
Il MIT, pagando abbonamenti a riviste scientifiche su cui pubblicavano i risultati delle proprie ricerche, presenta il concetto di Open Access: i risultati ottenuti da ricerche finanziate da denaro pubblico devono essere di pubblico dominio.
In questo modo i risultati di una ricerca sono fruibili da tutti.

\section{Pubblica Amministrazione.}
Un documento informatico è facilmente modificabile; dunque per garantire l'inviolabilità è utilizzato il sistema della \textit{firma digitale}, basato su un hash del documento crittografato in modo asimmetrico (la chiave pubblica è quella di decriptazione).
Il formato prediletto dalla Pubblica Amministrazione è il PDF/A, che a differenza del PDF standard non permette alcune operazioni sfruttabili da malintenzionati, ad esempio, non permettendo l'inserimento di caratteri nascosti e gif.
Inoltre il formato PDF garantisce l'indipendenza del documento dalla stampante in uso (cosa non verificata per i formati DOC e ODT); inoltre è aperto ma non modificabile.

\subsection{CAD.}
Il \textit{Codice di Amministrazione Digitale}, entrato in vigore all'inizio del 2005, stabilisce una serie di norme per la digitalizzazione della Pubblica Amministrazione.
Stabilisce i diritti dei cittadini nei confronti della pubblica amministrazione digitale; per farlo include la definizione di \textit{documento informatico}: questo non può essere modificato (o perlomeno deve essere possibile risalire al modificante).
La paternità delle modifiche ad un documento digitale è facilmente attribuibile grazie alla \textit{firma digitale}.

Il CAD definisce anche cosa si intende per \textit{duplicato informatico} (copia digitale di documento digitale) o di \textit{copia informatica di documento analogico} (documento digitale avente lo stesso contenuto di un documento cartaceo).

\subsection{PCT.}
Il \textit{Processo Civile Telematico} prevede che in tutti i processi di diritto civile lo scambio di documenti tra le parti avvenga in modo digitale (per ridurre il tempo di trasporto dei documenti cartacei). \newline

\subsection{SPID.}
Il \textit{Sistema Pubblico di Identità Digitale} è un progetto in tre fasi che sfrutta la firma elettronica per interagire con la Pubblica Amministrazione. Il progetto è diviso in tre fasi:
\begin{itemize}
\item Una prima fase in cui l'utente accede a servizi online con credenziali arbitrarie;
\item Una seconda fase in cui l'utente accede a servizi con credenziali arbitrarie e un codice temporaneo di accesso;
\item Una terza fase in cui l'utente dispone, oltre delle sue credenziali, di un supporto fisico per l'identificazione.
\end{itemize}

\subsection{PEC.}
La \textit{Posta Elettronica Certificata}, modellata sulla raccomandata analogica, garantisce l'identità della persona con cui si comunica e permette quindi l'interazione con la Pubblica Amministrazione: è considerato il \textit{domicilio digitale} della persona, anche per quanto riguarda violazione (rientra nella fattispecie delle violazioni di domicilio).
Quando una e-mail è inviata da una PEC a un'altra, l'SMTP server invia una conferma di ricezione e registra in automatico il log, sottoscritto da firma digitale, per inviarlo al mittente (senza passare dal destinatario, che quindi non può apportare modifiche).
La differenza, rispetto alla raccomandata, è che non è necessario il recepimento di una persona: si ha presunzione di conoscenza nel momento in cui l'email arriva nella casella di posta del destinatario.

\section{Sicurezza informatica.}
Nel 1984 si è inserita la prima \textit{backdoor} in un compilatore: questo compilava programmi inserendo altre backdoor; successivamente, nel 1995, è stata inserita una backdoor nel compilatore del compilatore: non si può più avere la garanzia che il proprio compilatore sia sicuro.
Da questo momento non è più possibile garantire la sicurezza informatica.

A inizio 2000 però si scropre che la maggior parte delle falle di sistema è dovuta ad errori umani: Mitmick (il primo \textit{cybercriminale}) afferma che spesso l'hacking è fatto non con competenze informatiche ma sfruttando il modo con cui gli utenti si interfacciano ad un sistema informatico (ingegneria sociale).
Esempio è il \textit{phishing}, ovvero la truffa tramite e-mail (si parla di \textit{spiral-phishing} se mirato verso una data persona). \newline

\subsection{Tipologie di attaccante.}
Fino al 2000 circa, i virus non avevano scopi malevoli ma servivano solamente per prendere in giro la vittima in contesti amichevoli; col tempo si sono individuate varie tipologie di \textit{hackers}, in base alle modalità e alle finalità dei loro attacchi.

\paragraph{\textit{Script-kid.}} Sono attaccanti senza scopo apparente, spesso con limitate capacità tecniche ma che usano strumenti già implementati.
\paragraph{Attaccanti politici.} Spesso gli attaccanti politici (come il movimento \textit{Anonymous}) non hanno grandi capacità tecniche ma sfruttano campagne di marketing per migliorare la propria immagine; se spinti da intenzioni malvage sono definiti anche \textit{cyber-terroristi}.
\paragraph{Industria del malware.} Esistono società altamente specializzate il cui oggetto sociale è la produzione di software malevolo su commissione.
Esempio è la BSA (Business Software Association), società russa opearativa nel biennio 2007-2008 e sparita senza lasciare tracce.
\paragraph{\textit{Squoters.}} Il loro unico obiettivo è fare danno; generalmente sono dotati di forti capacità tecniche.
\paragraph{\textit{Insiders.}} Sono lavoratori poco soddisfatti della propria condizione che provocano danni alla società o comunicano all'esterno informazioni riservate; è impossibile difendersi.
\paragraph{Hackers etici.} Eseguono \textit{pentesting} (anche non autorizzato) e comunicano a chi di dovere le falle di sicurezza trovate nel sistema; generalmente sono dotati di forti capacità tecniche.

\subsection{Firma elettronica.}
Negli anni '90 i processori sono sufficientemente potenti per implementare algoritmi di crittografia asimmetrica (PGP[\textbf{P}retty \textbf{G}ood\textbf{ P}rivacy]): la sicurezza non è data dalla segretezza dell'algoritmo, che è pubblico, ma dalla lunghezza e bontà della chiave.
È così rispettata la triade CIA:
\begin{itemize}
\item confidenzialità: le informazioni sono leggibili solamente da chi autorizzato;
\item integrità: non è possibile modificare i dati senza autorizzazione, e in ogni caso è tenuta traccia delle modifiche;
\item accessibilità: se autorizzati, è facile l'accesso ai dati.
\end{itemize}
È così permessa una sicurezza di tipo militare per il singolo cittadino.
L'algoritmo (RSM) è composto da una chiave privata, di decriptazione, e una pubblica, di cifratura: il documento è cifrato con la chiave pubblica del destinatario e inviato a questo che è l'unico in grado di leggere il file.

La Pubblica Amministrazione identifica più livelli di firma digitale:
\begin{itemize}
\item FE (Firma Elettronica): dati in forma elettronica acclusi o connessi logicamente ad altri dati elettronici per firmare il documento.
\item FEA (Firma Elettronica Avanzata): è connessa unicamente al firmatario ed è idonea ad identificarlo per ogni modifica effettuata nel documento;
\item FEQ (Firma Elettronica Qualificata): emessa su un dispositivo qualificato da Ente terzo certificato dallo Stato, ha lo stesso valore di una firma autografa.
\end{itemize}
Per garantire ulteriormente l'inviolabilità del file, un documento informatico della Pubblica Amministrazione contiene in calce l'hash del documento stesso (in due variazioni diverse), che è l'unica parte crittografata.

Una firma elettronica per poter essere considerata valida deve avere determinate caratteristiche:
\begin{itemize}
\item atenticità: la firma deve assicurare il destinatario della libera sottoscrizione del mittente;
\item non falsificabilità: la firma deve garantire la provenienza del documento in modo univoco;
\item non riusabilità: la firma non deve poter essere usata su altri documenti (garantito dall'algoritmo di hash);
\item non alterabilità: un documento firmato non può più essere alterato;
\item non contestabilità: il firmatario non può rinnegare la paternità della propria firma.
\end{itemize}

\subsection{NIS.}
La direttiva NIS, enrtata in vigore a Novembre 2018, regola il comportamento in caso di attacco a infrastrutture critiche.
La ratio è di evitare il ripetersi degli attacchi informatici che hanno colpito l'Estonia, facilitando anche opearzioni tra più Stati, creando un database centralizzato di attacchi informatici, popolato imponendo l'obbligo di segnalazione a determinati centri di interesse (CSTRI) che supportino l'attaccato.

La direttiva si applica ai fornitori di servizi essenziali (individuati dal Ministero dello Sviluppo Economico entro il 9/11/2018) per cui la fornitura del servizio necessita di un Sistema Informativo, e dunque eventuali danni provocano interruzioni della fornitura o danneggiamenti dell'utenza.
Inoltre, la direttiva si applica in automatico a fornitori nazionali o con rappresentante nazionale di servizi di \textit{e-commerce}, \textit{search engine} e \textit{cloud}.
Sono escluse le PMI (Piccole e Medie Imprese, cioè aziende con meno di 50 dipendenti o meno di 10mln di fatturato annuo).

La direttiva prevede obbligo di segnalazione e di implementare misure tecniche e organizzative di sicurezza proporzionali e adeguate.
Inoltre la direttiva impone la continuazione del servizio anche in caso di attacco informatico.
L'obbligo di notifica si applica solamente in caso di incidenti a impatto rilevante:
\begin{itemize}
\item sospensione del servizio per 5 milioni ore utente;
\item danni a persone tali da compromettere la vita;
\item danni per un valore economico pari a \euro{1.000.000} per un cittadino comunitario;
\item compromissione dei dati di 100.000 utenti.
\end{itemize}

La notifica deve avvenire senza giustificato ritardo, con pena pari compresa tra \euro{12.000} e \euro{120.000}, comunque contenuta rispetto a quella per la subita violazione di dati personali per insufficiente protezione.
La notifica deve comprendere inoltre le policy aziendali di sicurezza e la certificazione di autitor esterni sull'implementazione di tali misure	.

\section{GDPR.}
Secondo la nuova normativa, i dati sensibili tenuti da un'azienda devono essere crittografati (per garantire sia protezione che veridicità).

La regolamentazione della tecnologia, partendo da alcune assunzione di base, la tecnologia è qualunque cosa che si pone come un medium tra l’ambiente e il soggetto che si relazione con tale ambiente. A partire da determinate innovazioni tecnologiche si è sviluppata l’esigenza di tutelare la sfera di riservatezza che è la classica definizione di privacy. A fronte delle PIT (Privacy Invading Technologies) si è creata l’esigenza di tutela e regolamentazione della privacy, ma la risposta non è sempre stata una risposta sempre normativa infatti ci sono le cosiddette PET (Privacy Enhancing Technologies) delle tecnologie che consentivano di rispondere a queste esigenze di tutela con altri strumenti tecnologiche ad esempio la crittografia, anonimato ecc. Anzi spesso è questa risposta non legale ad essere più efficace dei metodi legali con magari delle sanzioni. Ormai la distinzione tra ambiente vs digitale e online vs offline non ha più senso visto l’ambiente in cui ci troviamo. \newline
Se trattiamo la privacy come concetto analogico e non digitale ( cioè la privacy non è solo 1 o 0 ma ha dei strati intermedi) quindi ha senso parlare di strumenti che ci permettono di aumentare il livello di protezione dei nostri dati. Il concetto classico della privacy fisica (tutela dall’essere visti in certe circostanze), ovviamente in un contesto digitale è quello della informational privacy, visto che le leggi trattano come privacy le sue informazioni. \newline
Le fasi del ciclo di informazione sono: 
a) apprensione $\rightarrow$ b) comunicazione $\rightarrow$ c) registrazione o archiviazione $\rightarrow$ d) telecomunicazione (trasmissione) $\rightarrow$ e) manipolazione \newline
Rispetto a questi step ci sono stati vari sviluppi di tutela, ad esempio dall’ultima fase della manipolazione è nata la data protection. Nel 1890 il primo step in un articolo di Warren e Brandeis in cui si parla del diritto alla privacy, dice che non si possono diffondere le foto senza il consenso dell’individuo, era dovuto alla nascita delle telecamere portatili e della giornale scandalistica, che è uno dei primi oggetti che ha fatto sorgere l’esigenza della privacy.\newline
Un altro step importante nel 1928 con il tema dell’intercettazione delle conversazione dentro le case, la polizia usa strumenti per captare le informazioni (conversazioni) dentro casa di uno spacciatore (il signor Olmstead) senza chiedere autorizzazione del magistratura (è una forma di trespassing ma allora non c’era nessuna legge a riguardo). La corte Suprema da ragione alla polizia poiché non erano mai entrati dentro la proprietà privata quindi nessuna violazione della privacy fisica (ATTENZIONE non c’era ancora informational privacy). L’avvocato Brandeis firma un dissenso dicendo non era d’accordo con il fatto di aver dato ragione alla polizia, dicendo che la legge non era formata in base alla tecnologie che avevano in quel momento altrimenti vi sarebbe stato una legge che avrebbe richiesto il permesso del magistrato per proteggere la privacy del cittadino americano. \newline
Nel 1970 la privacy nel senso tradizionale piano piano si sviluppa a partire da USA, nel 1970 vediamo a livello mondiale la prima forma di regolamentazione dei dati personali, qua la PIT è il computer mainframe. Il problema è dovuto al fatto che non ci si fida ancora del governo tedesco e svedese (per la WWII) e quindi c’è la necessita di regolamentare i dati personali dovute ai censimenti per l’uso che ne potrebbe fare il governo. Le esigenze sono in pratica 2:
\begin{itemize}
	\item Impedire la nascita di banche dati segrete, rendere informazione per far sapere al soggetto a cui i dati si riferiscono il perché, la durata del trattamento dei dati.
	\item Impedire usi secondari di questi dati, se raccogli i dati per un motivo non devi usare i dati per altre finalità.
\end{itemize}
Queste due esigenze anche oggi sono alla base della data protection. Oggi il caso più famoso della concretizzazione di questi due problemi è quello di Cambridge Analitica. \newline
Differenze nella privacy nel senso tradizionale e data protection:
\begin{itemize}
	\item     • Privacy: Individuo non deve essere sottoposto a ingerenze da parte del pubblico, la sfera ecc. \newline
	Privacy nel senso tradizionale: Sottrazione del individuo rispetto al contesto.
	\item Data Protection: Rapporto di fiducia tra uno che gestisce un database (data collector) e l’individuo (data subject) a cui le informazioni si riferiscono, se nella privacy parlavamo della sottrazione qua si parla della limitazioni dell’uso dei dati, può essere definita come una sorte di attrito nella circolazione delle informazione. La privacy (informational) è una sorte di controllo in questo ambito, in particolare la privacy viene definita come la possibilità di decidere in maniera autonoma quando, come e in che misura comunicare ad altri informazione che li riguardano. 
\end{itemize}

L’ultimo step è nel 1994, La copertina di Time dedica la copertina ad internet, la data protection resta un questione legata a data controller e data subject. Qua la banca dati non è più in mano al governo come nel 1970 ma ad internet, quindi il fenomeno di violazione dei dati possono diventare un fenomeno di massa e gli attori principali sono le corporation private. \newline
Nel GDPR la norma che riguarda la sicurezza informatica (Art. 32) ha come primo riferimento, quindi come primo strumento di tutela che il titolare del trattamento (il data controller) deve implementare, quello di incrementare strumenti volti all’anonimizzazione (pseudonimizzazione) e cifratura dei dati personali. Con il GDPR si eliminano le misure minime di sicurezza, essendo un elenco specifico a cui era collegato la sanzione penale dovevano essere tassative, ma con il periodo dovrebbero essere modificate, e per una norma a lunga durata diventerebbero obsolescenti e insicure. E’ rimasto il fatto di avere le misure di sicurezza idonee ed adeguate, che sono state specificati almeno 4 ambiti (tra cui anche quello di cifratura dei dati), che rappresentano punti di riferimento che il titolare del trattamento deve tenere in considerazione. I 4 ambiti sono:
\begin{enumerate}
	\item Cifratura dei dati
	\item Utilizzo di sistemi di sicurezza perimetrale quindi avere l’antivirus aggiornato ecc.
	\item Procedura Specifica in maniera informatica per garantire in tempi certi la disponibilità e l'accessibilità dei dati.
	\item La verifica dei test (penetration test) a cadenza regolari per valutare il livello di sicurezza della propria azienda, per garantire i 3 punti appena elencati.
\end{enumerate}
Elemento fondamentale di questa norma è che tutti i soggetti bene e servizi ai cittadini europei sono tenuti a rispettare il GDPR. Il GDPR tiene in mente i grandi colossi come Facebook, Google ecc. Il nuovo regolamento si basa sull’accountability, fondamentalmente responsabilizzazione e rendicontazione, al data controller sono imposti tutta una serie di oneri che sostituiscono autorizzazioni, in buona sostanza deve predisporre un documento che dovrà descrivere in particolare il tipo di trattamento e dovrà descrivere le misure di sicurezza che ha adottato per limitare il rischio di violazione dei dati personali. Questo documento sarà la policy per tale tipo di trattamento. Il momento in cui si dovesse verificare un problema. In questa normativa viene fornita una cornice che verrà dopo riempita dalle regole tecniche, best practice e formazione specifica da parte dei operatori in modo tale da avere come rifermento con la norma ma patrimonio di conoscenze di un particolare settore. Il 3$^o$ aspetto da considerare è quello di sanzioni, le sanzioni previste sono due tipologie di sanzioni amministrazione con un massimale di 20 milioni per una categoria e 10 milioni per un’altra categoria di violazioni, nelle quali se 20 milioni di euro dovessero essere considerati poco si può arrivare al 4\% del fatturato annuo dell’anno precedente. \newline
Il GDPR è applicabile dal 25 maggio 2018, prima di questa data tutti gli stati membri hanno avuto 2 anni dal maggio 2016 per emanare delle norme di coordinamento con la norma nazionale e quella europea. In Italia il d.lgs. 101/2018 (per il coordinamento) è entrato in vigore il 19 settembre 2018. 
I 6 punti fondamentali della data protection sono:
\begin{enumerate}
	\item Fondamenti di liceità del trattamento: Ogni trattamento del dato personale che non sia svolto per motivi personali, deve basarsi su un fondamento di liceità (e quindi avere una base giuridica).
	\item Informativa: I contenuti dell’informativa sono elencati in modo tassativo nel regolamento e in parte sono più ampi rispetto al Codice. Nell’informativa ci dovrà essere il titolare del trattamento, la finalità del trattamento e la base giuridica del trattamento.
	\item Diritti degli interessati: Il titolare deve dare entro 1 mese dalla richiesta una risposta all’interessato e i diritti sono:
	\begin{itemize}
		\item Diritto di accesso
		\item Diritto di rettifica
		\item Diritto di cancellazione (nuovo)
		\item Diritto di limitazione
		\item Diritto alla portabilità (nuovo)
		\item Diritto di opposizione
		\item Diritto di revocare il consenso
	\end{itemize}
	\item Titolare, responsabile e incaricato
	\item Approccio basato sul rischio e misure di accountability
	\item Trasferimenti di dati verso paesi terzi
\end{enumerate}

La normativa, che in generale si applica solo alle persone fisiche (l’interessato), per quanto riguarda l’attività promozionale sia applica sia alle persone fisiche che alle persone giuridiche. Un dato non deve essere considerato pubblico solo se facilmente reperibile su internet. Un dato è pubblico se viene da determinate tipi di fonti che lo codificano come tale. Potrebbe essere che uno li abbia pubblicati per un altro motivo e io non posso cambiarne la finalità solo perché disponibile online. Questa sorta di raccolta dei dati ad esempio per arricchire i database è vietata dalla normativa europea e italiana. La normativa del 2013 individua una serie di modalità volta alle attività promozionale rispetto a queste modalità stabilisce e sceglie se esiste un obbligo di consenso preventivo (opt-in) o il di consenso preventivo – eccezione (opt-out) (posso oppormi al trattamento) 
Gli aspetti dell’attività promozionale sono due: 

\begin{itemize}
	\item L’informativa che specifica non solo l’attività promozionale ma anche la modalità di come questa attività promozionale viene raggiunta, quindi se ti contattano tramite il telefono, mail ecc. 
	\item Rispetto a tale modalità ricevere il consenso se necessario dell’interessato. Il consenso è la seconda modalità, in questi casi parla di opt-in.
\end{itemize}
Quindi non mi è consentito mandare le mail esplorative per richiedere il consenso se non l’ho a monte. Ma è possibile contattare telefonicamente mediante un operatore fisico per ricevere il consenso del contraente per ricevere comunicazione promozionale, ma si ha la possibilità di disiscriversi (opt-out) ad esempio iscrivendosi al registro delle opposizioni, un altro ambito dvoe si può esercitare il diritto all’opt-out è quello di marketing postale.\newline
Il consenso deve essere staccato rispetto all’erogazione del servizio, rispetto alla raccolta di altri dati, deve essere libero e si deve basare sull’informativa. Queste caratteristiche sono rimaste anche nel GDPR. \newline
Oltre al opt-in, opt-out esiste un terzo regime soft-spam in cui trattiamo i dati del nostro cliente, il cliente nel contratto ci lascia l’email o indirizzo postale, e prevede la possibilità di usare email o indirizzo per inviare comunicazioni di tipo promozionale o commerciale.\newline
Per i dati sanitari invece, prima del GDPR c’era il divieto di utilizzare i dati sanitari salvo un autorizzazione del Garante, un informativa, consenso scritto e consenso scritto al trasferimento verso paesi terzi (in caso ce ne fosse bisogno). Il GDPR ha ribadito il fatto che il dato non è utilizzabile se non da un autorizzazione dell’interessato e non fosse per un uso pubblico, non è possibile farne uso promozionale, di profilazione o qualunque non consentito dall’utente. I dati della regione Lombardia potevano essere usati dopo l’anonimizzazione da un operatore. Le norme di sicurezza sono delle norme specifiche, devono garantire la segregazione dei dati, garantire i vari livelli di accessibilità dei dati in modo che non tutti possano accedere a tali dati e garantire la cifrature e copertura dei dati e garantire la data-recovery e comunicazione dei breech che hanno ricevuto. \newline
La privacy su internet, l’elemento fondamentale è il cookie, i cookie sono divisibili in due macro categorie: cookie di prima parte (di tipo tecnico) e di terza parte (di profilazioni). Il 99\% dei cokie utilizzati sono di terza parte. In mezzo ai due tipi di cookie c’è un terzo tipo quello di cookie analitici che permettono di vedere le statistiche riguardante le pagine, e questi possono essere usati senza adempiere una serie di obblighi solo se sono anonimizzati. Per l’uso dei servizi cloud regole particolari non ne sono state introdotte, sta al cliente di scegliere il cloud che possa fornire un certo livello di sicurezza.\newline

\subsection{\textit{Cyber Act.}}
L'Organo Europeo ENISA, entro fine anno, stabilirà una bozza di regolamento (quindi valido senza recepimento da parte dei Paesi membri) per la certificazione di oggetti IoT e simili.

\section{The Cathedral and the Bazaar}
L'autore prima credeva in un visione alla cattedrale, cioè ogni software
viene attentamente lavorato senza alcuna versione beta e quando dovesse
uscire doveva essere perfetto. Rimane stupefatto dal modo di fare di
Linus Torvalds, questo modo sembra caotico come il bazaar. L'autore
rimane stupefatto di come (e come mai) questo modo alla bazaar
funzionasse bene. E per capirlo provò a fare un progetto open source
seguendo questa visione.

Il progetto è sulla mail forwarding, lui non poteva farla inoltrare dal
proprio PC personale a quello del lavoro visto che IP del PC non è
statico, allora pensa di usare un POP3 client, ne trova diversi su
internet ma tutti hanno qualche problema, anzi il problema in comune e
che non vi era modo di fare reply perché le email non venivano
registrate ma solo la parte iniziale prima di @. Allora con l'ideologia
che ``Ogni buon software inizia dalla frenesia personale di uno
sviluppatore.'', pensò di fare un POP client per le mail. Ma sfruttando
la lezione: ``I bravi programmatori sanno cosa scrivere. I migliori
sanno cosa riscrivere (o riusare).'' scelse un client POP3 più vicino a
ciò che lui voleva e cominciò a lavorarci sopra. E' possibile che nel
corso dello sviluppo, il codice iniziale scompaia totalmente, ma è
sempre bene iniziare a scrivere il codice a partire da qualcosa. Non è
neanche semplice da cosa partire ad esempio l'autore prima scelse un
modello di un Coreano, ma più tardi si imbatté in un altro client che
gli sembrava più utile per le sue esigenze. La decisione di cambiare la
base può essere difficile, perché si perde tutto il lavoro fatto fino ad
allora, ma bisognerà farlo come di qualcuno ``Preparati a buttarne via
uno, dovrai farlo comunque.'' L'autore del secondo client di nome
popclient, ma il suo autore aveva perso interesse in questo programma, e
``Quando hai perso interesse in un programma, l'ultimo tuo dovere è
passarlo a un successore competente'' e ``Se hai l'atteggiamento giusto,
saranno i problemi interessanti a trovare te.'', infatti, in questo caso
l'autore di popcliente diede all'autore il suo codice. Insieme al codice
ereditò anche i suoi utenti e secondo l'ideologia di Linus i utenti si
possono trasformare in co-sviluppatori anzi è la strada migliore per un
rapido miglioramento del codice e per un debugging efficace. Il pensiero
è rappresentato con ``Distribuisci presto. Distribuisci spesso. E presta
ascolto agli utenti.'', un punto cruciale che permetteva a Linus di
aggiornare il suo kernel anche diverse volte al giorno era internet e la
sua di rete di co-sviluppatori. E da qui la legge di Linus: ``Dato un
numero sufficiente di occhi, tutti i bug vengono a galla.''. Un altro
motivo per il successo della visione bazaar è che secondo la visione
cattedrale, l'utente dopo aver aspettato anche 6 mesi trova una versione
incompleta e ne rimane insoddisfatto, mentre nella visione bazaar viene
ricompensato con il fatto che la sua voce sulle imperfezioni vengono
preso aggiustate. L'autore per il suo progetto seguì tutto ciò che era
stato detto da Linus con una ricompensa immediata, grazie alle critiche
e avvolte anche soluzioni da parte dei suoi beta tester. Il periodo di
svolta ci fu quando un suo beta tester gli mandò il codice di SMTP
forwarding, da qui ebbe l'idea di come risolvere/migliorare i suoi
problemi e cambiò anche la prospettiva della sua domanda e giunse a una
risposta, a questo punto cambiò anche il nome in fetchmail del suo
client.

Le pre-condizioni necessarie per lo stile bazaar: Innanzitutto, ci deve
essere il materiale (del codice sorgente), a disposizione su cui
iniziare a lavorare e dedurre dai propri utenti idee buone. Un altro
fatto è la comunicazione con i propri co-sviluppatori, senza la quale
non sarebbe possibile tutto lo sviluppo. Secondo l'idea di Brooks,
aggiungendo altri sviluppatori la complessità della rete diventa
esponenziale e diventa impossibile comunicare tra di loro, e ciò ritarda
sempre di più il progetto. Ma ciò non può essere vero, altrimenti Linus
non sarebbe mai diventato ciò che è. Brooks viene corretto da Gerald che
spiega come ciò sia possibile se si programma senza ego. Questo modello
non si sviluppò prima per una questione anche legale sulle licenze, e il
fatto che internet non fosse così sviluppato. Questa programmazione
senza ego avveniva solo nelle in comunità grandi come MIT, UC Berkley
ecc., dov'è facile trovare persone in gamba disposte ad aiutare. Ma
bisogna avere anche un certo stile di leadership per tenere insieme la
squadra.

\section{Cultural and Social Factors in Cybersecurity}

\textbf{Le skill della cybersecurity nelle minacce odierne.}

Dopo il 2015 tante università riconoscono l'utilità della parte
umanitaria della cybersecurity, creando delle facoltà apposta per queste
materie. In un sondaggio, che esaminava le skill necessarie per condurre
le operazioni di cybersecurity, ci furono 3 casi, venivano poste domande
riguardante la rilevanza tecnica e umanitaria, e come risultato si
ebbero 5 skill tecniche e 5 umanitarie, queste 10 erano precedentemente
già state individuate in un'altra ricerca. E come conclusioni giunsero
alle seguente idee:

\begin{itemize}
	\item
	Bisogna cercare nuovi approcci alla cybersecurity, non basati solo su
	computer science ma anche aspetti umanitari.
	\item
	Includere approcci come sociologia, psicologia per capire i
	comportamenti maliziosi in una maniera più efficace.
	\item
	Forzare una cooperazione tra programmi educazionali diversi e
	approcciare la tecnologia in favore dell'umanità.
\end{itemize}

Le KSAs (Knowledge, Skills and Abilities), più pertinenti furono,
iterazione macchina-uomo, psicologia criminologia, comportamento
sociologico, e comportamento umano. Si analizzarono 10 lavori più vicini
a cybersecurity e si analizzarono i loro KSA e si rimuovono i KSA
ridondanti. Per ciascun KSA trovato si considerò l'opinione di un cyber
analista nel campo del lavoro, ad esempio le skill nelle macchine
virtuali sono più utili di un disassemblatore di PC. Facendo così si
aggiungono altre KSA utile per il futuro sopratutto.

Uno dei punti cruciali per capire la cybersecurity oggi è capire i suoi
bisogni da parte di altre materie e con l'aumento dei attacchi cyber
diventa sempre più cruciale capirlo, le facoltà di cybersecurity non
riescono a soddisfarli e quindi si crea un gap tra ciò che è richiesto
al lavoro e ciò che uno ha imparato all'università. Altri modifiche
verranno fatte al questionario per riempire il gap nei bisogni della
cybersecurity.

\textbf{Il ruolo di fattori umani nella cybersecurity.}

Nei cyber attacchi, che stanno aumentando sempre di più, il rischio più
grande è umano stesso o meglio le sue vulnerabilità. Bisognerebbe
insegnare a loro cosa fare e cosa no aumentando la cultura della
cybersecurity. Il questionario è composto da 15 domande appartenenti
all'ambito:

\begin{enumerate}
	\item
	Fattori individuali riguardanti: percezione, valutazione e gestione
	del rischio
	\item
	IT e cyber rischio
	\item
	Rischio nella vita privata e in organizzazione
	\item
	Comportamenti e misure di sicurezza
\end{enumerate}

Analisi quantitativa:

Nel 1\textsuperscript{o} caso si nota che tra i fattori che influenzano
la percezione spicca ``l'abilità di poterlo controllare'' seguito da
``interesse riguardante l'oggetto'' e ``Le emozioni coinvolte''. Mentre
per la valutazione abbiamo che vince ``E' difficile da controllare''
seguito dal ``numero di individui coinvolti''.

Nel 2\textsuperscript{o} caso il ruolo dei social abbiamo il problema
del rischio alla privacy individuale, seguito dal potenziale di
incontrare persone sgradevoli.

Nel 3\textsuperscript{o} caso nella rischio nella vita privata abbiamo
il problema del furto d'identità seguito dalle frodi e malware mentre
nelle organizzazioni abbiamo il problema della perdita di informazione,
frodi interne e attacchi critici contro le infrastrutture (cyberattack).

Nel 4\textsuperscript{o} caso invece per la prevenzione vediamo
prevalere ``analisi del rischio'' e ``aggiornamento e formazione''
seguito da comunicazioni interne e tecnologie avanzate. Mentre su una
scala da 1-5 sull'importanza dei parte del sistema abbiamo: Persone
(4,43), tecnologie (4,23), procedure e risorse (4,05) e infine
regolazioni (3,63).

Analisi qualitativa:

Usando come keywords: attitude, relationship, behavior e trust, e usando
dei text analysis si ha che la parola attitude viene collegata a
positivo 12\% delle volte seguito da responsabilità (5\%) e
collaborazione (4\%). Mentre per la parola relationship abbiamo:
comunicazione (18\%), cooperazione(6\%) e sharing(4\%). Behavior:
correttezza (11\%), responsabilità (7\%). Trust: sicurezza(10\%),
reciprocità (6\%) e affidabilità (4\%).

Concludiamo che una cultura cybersecurity non può essere costruito senza
umani e la loro conoscenza riguardanti i rischi. Ma è fondamentale
un'adeguata formazione sopratutto nel campo pratico e nelle capacità
comunicative.

\textbf{Un percorso per migliorare l'awareness cybersecurity per le
	facoltà ingegneristiche}.

La crescente tecnologia in Germania mira a unire tante tecnologie in
maniera gerarchica e quindi un cooperazione tra di loro. In questo caso
è importante cercare le vulnerabilità sia software che hardware. Ma
comunque la maggior parte di problemi di cybersecurity è dovuto agli
errori umani e la loro mancanza di conoscenza. In uno studio i studenti
devono leggere il report di altri studenti e dare un loro feedback.

Il mondo digitale è senza frontiere quindi l'espansione è globale e
questo porta ad altre vulnerabilità. In questo caso anche un semplice
trasferimento di file può generare un attacco, quindi si ricorre alle
crittografie, investendo più risorse. Nelle compagnie più grandi il data
leak potrebbe presentarsi come un grosso problema, quindi per loro è un
investimento usare strumenti di crittografia più robusti. Ma comunque la
formazione del personale rimane cruciale. Vogliono introdurre nelle
università di ingegneria la materia di crittografia, le regole più
cruciali sono: La motivazione dell'insegnante sopratutto perché dovrà
spiegarlo a persone che non hanno una formazione della materia, bisogna
chiare l'obiettivo e il programma del corso. E queste cose in realtà
sono una catena, una segue l'altra ma senza un pezzo non ha senso.
L'obiettivo del corso deve essere portare lo studente a :

\begin{itemize}
	\item
	Identificare le sfide tecniche, sociale e politiche delle industrie
	(4.0)
	\item
	Spiegazione delle tecnologie e metodi della cybersecurity.
	\item
	Spiegazione e chiarificazione delle contromisure per proteggere nelle
	industrie 4.0
	\item
	Motivazione per auto-imparare contenuti complessi
	\item
	Acquisizione delle skill sociali per lavorare in squadra e skill delle
	presentazioni
	\item
	Imparare a dare un feedback costruttivo e semantico
\end{itemize}

L'esame consisterà in un progetto in team, dare un feedback a un altro
team e infine un'esame orale sul progetto e sul corso.

\textbf{Fattori di cittadini ingegneri sociali in situazioni critiche}.
(Paper in Estonia 2014)

Ingegneria sociale è sempre esistita, usare le fake news per cambiare
opinione riguardante le politiche, usando questa ingegneria diversi
movimenti sono stati effettuati nel mondo odierno, quindi non deve
essere sottovalutato, ad esempio nelle elezioni possono essere molto
influenzanti.

Le norme solitamente si concentrano su cyber crimine o sviluppo delle
difese strategiche senza concentrarsi sulla cultura informazionale come
Estonia. Per quanto riguarda lo studio, diversi paesi di UE hanno
diverse ideologie, spesso dividono gli argomenti di cybersecurity in
sotto-ambiti che finiscono per essere insegnati da professori di
categorie completamente diverse e in diversi corsi, Estonia invece ha
adottato una nuova soluzione di insegnarlo passo a passo in base a
gradi:

\begin{enumerate}
	\def\labelenumi{\arabic{enumi}.}
	\item
	Grado 1-3, Sicurezza digitale: Capire la differenza tra il mondo
	online ed offline, capire cos'è una password sicura, i regolamenti
	della privacy, cyberbullismo, applicazione e in-app a pagamento e
	gratuiti, sovrauso della tecnologia e i rischi alla salute.
	\item
	Grado 4-6, Igiene digitale: Capire le norme di comportamento online,
	sapere e valutare le fonti, proteggere la propria identità digitale e
	strumenti (da virus, spyware), risolvere problemi basilari riguardante
	la tecnologie e reti.
	\item
	Grado 7-9, Igiene cyber: Distinzione tra comportamento legale e
	illegale, capire le norme, regolamenti e legge criminologica, saper
	spiegare gli effetti della tecnologia sulla vita. Analizza e usa la
	corretta terminologia di cybersecurity per risolvere i problemi.
	Controlla a casa di aver migliorato la situazione seguendo i
	suggerimenti. 
	\item
	Palestra, Difesa cyber: E' fedele al proprio paese, ha atteggiamento
	giusto per proteggere il paesi e gli alleati, ha valori democratici ed
	è responsabile delle proprie azioni. Ha le skill necessarie per
	proteggere la propria casa da minacce cyber basilari. Capisce le norme
	e regolamenti di Estonia riguardante la sicurezza ed e-governo.
	Analizza data e informazione criticamente. Sa come spiegare,
	documentare e condividere l'attacco cyber agli ufficiali e sa usare le
	proprie skill e competenze per scegliere una professione futura.
\end{enumerate}

Fanno un sondaggio per i metodi per risolvere i problemi di
manipolazione online, saper predire quando l'infrastruttura potrebbe
fallire durante un attacco.

Dai risultati abbiamo che per risolvere il problema di manipolazione
online, nei casi più semplici basta analizzare il problema e sentire la
puzza. Nel sondaggio la prima cosa che fecero è cercare se un'altra
fonte dicesse una cosa simile, poi hanno discusso tra di loro e come
3\textsuperscript{o} step hanno voluto chiamare la polizia. Si nota che
sono spesso i giovani a chiamare subito la polizia. La maggior parte in
generale cercò di leggere la storia e commenti per capire se è falsa o
meno, come seconda opzione parlare con qualcuno più conoscente di loro
per chiedere conferma. Quelli a cascarci sono spesso studenti.

Per i fattori determinati in situazioni critiche (collasso del sistema
economico digitale ad es Estonia nel 2007) vediamo che durante:

\begin{itemize}
	\item
	Estate, le persone si disconnettono e vanno alla spiaggia o a fare la
	spesa, e se la situazione non si è risolta in 2 settimane, cercano
	altre opzioni come lasciare il paese o commettere crimini. Ma in
	generale rimangono rilassati.
	\item
	In autunno o primavera, c'è meno disconnessione, sopratutto i giovani
	cominciano a pressare per la risoluzione della situazione
	urgentemente, mentre le persone che lavorano o i genitori sono più
	rilassati.
	\item
	In inverno invece stanno più a casa per il freddo e la stocca di cibo
	che c'è, se c'è indisponibilità cominciano a rubare spesso i giovani
	maschi, gli adulti lasciano il paese dopo 7 giorni, seguiti da altri.
\end{itemize}

Per la responsabilità di awareness si nota che le femmine sono più
disposte dei maschi, la maggior parte delle persone credono che la
formazione debba avvenire nelle scuole e il modo meno efficace sia
utilizzando i media tradizionali. Mentre gli adulti per loro hanno
pensato come soluzione i media online. Gli Estoni si fidano più del
governo riguardante questo ambito.

\end{document}
