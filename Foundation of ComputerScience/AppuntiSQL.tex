\documentclass[a4page, 11pt]{article}

\usepackage{graphicx}
\graphicspath{ {./img/} }
\usepackage[margin=0.65in]{geometry}
\usepackage[utf8]{inputenc}
\usepackage[english]{babel}
\usepackage{enumitem}
\usepackage{booktabs}

%`` for quoting

\title{Foundations of Computer Science  - SQL}
\author{}
\date{}

\begin{document}

\maketitle

\section{Introduction to Relational DataBases}

The relational DB are organized in tables that can be linked together. A table in the relational model represents a relation. For example in the following database each row describes a single book:

\begin{table}[h]
	\centering
	\begin{tabular}{|l|l|}
		\hline
		
		\textbf{ISBN} & \textbf{Title}\\ 
		\hline
		978-1-449-30321-1 & Scaling MongoDB\\ 
		\hline
		978-1-491-93200-1 & Graph Databases\\ 
		\hline
		978-1-449-39041-9 & Cassandra: The Definitive Guide\\ 
		\hline
		007-709500-6 & Database Systems\\
		\hline
		
	\end{tabular}
	\caption{Some sample books}
\end{table}

The data in a relational model is organized into columns and each entry(cell) contains a \textbf{SINGLE} piece of data, so the main problem of relational model is to handle data with multiple elements in a single column for example a book with two authors.

The solution proposed by the relational model is to link two tables and to do so we need an identifier ID for each row.
Here is an example of two linked tables.

\begin{table}[h]
	\centering
	\begin{tabular}{|l|l|l|}
		\hline
		
		\textbf{book\_id} & \textbf{ISBN} & \textbf{Title}\\ 
		\hline
		1 & 978-1-449-30321-1 & Scaling MongoDB\\ 
		\hline
		2 & 978-1-491-93200-1 & Graph Databases\\ 
		\hline
		3 & 978-1-449-39041-9 & Cassandra: The Definitive Guide\\ 
		\hline
		4 & 007-709500-6 & Database Systems\\
		\hline
		
	\end{tabular}
	\caption{Books Table with ID}
\end{table}

\begin{table}[h]
	\centering
	\begin{tabular}{|l|l|l|}
		\hline
		
		\textbf{author\_id} & \textbf{Name} & \textbf{Surname}\\ 
		\hline
		1 & Ian & Robinson\\
		\hline
		2 & Kristina & Chodorow\\ 
		\hline
		3 & Riccardo & Torlone\\
		\hline
		4 & Paolo & Atzeni\\
		\hline
		5 & Stefano & Ceri\\
		\hline
		6 & Stefano & Paraboschi\\
		\hline
		7 & Eben & Hewitt\\
		\hline
		
	\end{tabular}
	\caption{Authors Table with ID}
\end{table}

To connect these two tables we use a ``relation'' or ``join'' table. In the example the ``join'' table is the following:

\begin{table}[h]
	\centering
	\begin{tabular}{|l|l|}
		\hline
		
		\textbf{book\_id} & \textbf{author\_id} \\
		\hline
		2 & 1\\
		\hline
		3 & 7\\
		\hline
		1 & 2\\
		\hline
		4 & 4\\
		\hline
		4 & 5\\
		\hline
		4 & 6\\
		\hline
		4 & 3\\
		\hline
	\end{tabular}
	\caption{BookAuthors Table}
\end{table}

Here we can see the need of using an ID to identify each row.


\end{document}
