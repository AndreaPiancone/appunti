\documentclass[a4page, 11pt]{article}

\usepackage{graphicx}
\graphicspath{ {./img/} }
\usepackage[margin=0.65in]{geometry}
\usepackage[utf8]{inputenc}
\usepackage[english]{babel}
\usepackage{enumitem}
\usepackage{booktabs}

\usepackage{listings}
\usepackage{color}

\definecolor{dkgreen}{rgb}{0,0.6,0}
\definecolor{gray}{rgb}{0.5,0.5,0.5}
\definecolor{mauve}{rgb}{0.58,0,0.82}

\lstset{frame=tb,
	language=SQL,
	aboveskip=3mm,
	belowskip=3mm,
	showstringspaces=false,
	columns=flexible,
	basicstyle={\small\ttfamily},
	numbers=none,
	numberstyle=\tiny\color{gray},
	keywordstyle=\color{blue},
	commentstyle=\color{dkgreen},
	stringstyle=\color{mauve},
	breaklines=true,
	breakatwhitespace=true,
	tabsize=3
}

%`` for quoting

\title{Foundations of Computer Science  - SQL}
\author{}
\date{}

\begin{document}

\maketitle

\section{Introduction to Relational DataBases}

The relational DB are organized in tables that can be linked together. A table in the relational model represents a relation. For example in the following database each row describes a single book:

\begin{table}[h]
	\centering
	\begin{tabular}{|l|l|}
		\hline
		
		\textbf{ISBN} & \textbf{Title}\\ 
		\hline
		978-1-449-30321-1 & Scaling MongoDB\\ 
		\hline
		978-1-491-93200-1 & Graph Databases\\ 
		\hline
		978-1-449-39041-9 & Cassandra: The Definitive Guide\\ 
		\hline
		007-709500-6 & Database Systems\\
		\hline
		
	\end{tabular}
	\caption{Some sample books}
\end{table}

The data in a relational model is organized into columns and each entry(cell) contains a \textbf{SINGLE} piece of data, so the main problem of relational model is to handle data with multiple elements in a single column for example a book with two authors.

The solution proposed by the relational model is to link two tables and to do so we need an identifier ID for each row.
Here is an example of two linked tables.

\begin{table}[h]
	\centering
	\begin{tabular}{|l|l|l|}
		\hline
		
		\textbf{book\_id} & \textbf{ISBN} & \textbf{Title}\\ 
		\hline
		1 & 978-1-449-30321-1 & Scaling MongoDB\\ 
		\hline
		2 & 978-1-491-93200-1 & Graph Databases\\ 
		\hline
		3 & 978-1-449-39041-9 & Cassandra: The Definitive Guide\\ 
		\hline
		4 & 007-709500-6 & Database Systems\\
		\hline
		
	\end{tabular}
	\caption{Books Table with ID}
\end{table}

\begin{table}[h]
	\centering
	\begin{tabular}{|l|l|l|}
		\hline
		
		\textbf{author\_id} & \textbf{Name} & \textbf{Surname}\\ 
		\hline
		1 & Ian & Robinson\\
		\hline
		2 & Kristina & Chodorow\\ 
		\hline
		3 & Riccardo & Torlone\\
		\hline
		4 & Paolo & Atzeni\\
		\hline
		5 & Stefano & Ceri\\
		\hline
		6 & Stefano & Paraboschi\\
		\hline
		7 & Eben & Hewitt\\
		\hline
		
	\end{tabular}
	\caption{Authors Table with ID}
\end{table}

To connect these two tables we use a ``relation'' or ``join'' table. In the example the ``join'' table is the following:
\pagebreak
\begin{table}[h]
	\centering
	\begin{tabular}{|l|l|}
		\hline
		
		\textbf{book\_id} & \textbf{author\_id} \\
		\hline
		2 & 1\\
		\hline
		3 & 7\\
		\hline
		1 & 2\\
		\hline
		4 & 4\\
		\hline
		4 & 5\\
		\hline
		4 & 6\\
		\hline
		4 & 3\\
		\hline
	\end{tabular}
	\caption{BookAuthors Table}
	\label{table:join}
\end{table}

In the Table ~\ref{table:join} we can see the need of using an ID to identify each row, so it can be referenced uniquely. 
In some cases we might not need a third table (join table): one of these cases is the ``One-to-Many" relationship. 
The One-to-Many relationship also requires that rows in tables have unique IDs, but unlike the join table used in Many-to-Many relationship, the table with the many side of data has a column reserved for the IDs of the one side of data.

The IDs used to uniquely identify the rows described in the tables are called ``Primary Keys''(PK). 
If this primary key is used in another table it is called ``Foreign Key''(FK) and it is usually not unique in the new table. The purpose of the foreign key is to link the two tables in one-to-many relationship.

Usually the primary key generation process is left to the RDBMS (it's safer this way), which automatically generates the key and usually it is an ordinary integer.
For the join tables, the primary key is a combination of the foreign keys. A primary key comprised of more than one attribute is called a ``Composite Primary key''(CPK).

\section{Introduction To SQL}
SQL stands for Structured Query Language and it is a standard language for querying and manipulating data. It is a very high-level programming language and it is very well optimized.


SQL is a:
\begin{itemize}[noitemsep]
	\item Data Definition Language (DDL):\newline
	 It defines relation \textit{schema} and creates/alters/deletes tables and their attributes.
	\item Data Manipulation Language (DML):\newline
	It Inserts/deletes/modifies tuples in tables and queries one or more tables.
\end{itemize}

A \textbf{relation} (or \textbf{table}) in SQL is a multiset of tuples having the attributes specified by the schema. Where a \textbf{multiset} is an unordered list (so multiple duplicates instances are allowed) and an \textbf{attribute} is a typed data entry present in each tuples in the relation and it must have an atomic type in standard SQL. The atomic types are:
\begin{itemize}
	\item Characters: CHAR(20), VARCHAR(50)
	\item Numbers: INT, BIGINT, SMALLINT, FLOAT
	\item Others: MONEY, DATETIME, etc.
\end{itemize}
A \textbf{tuple (row)} is a single entry in the table having the attributes specified by the schema.

The \textbf{schema} of a table is the table name, it's attributes and they types, a key is an attribute whose values are unique. A \textbf{key} is a minimal subset of attributes that acts as a unique identifier for tuples in a relation. A key is an implicit constraint on which tuples can be in the relation: so if two tuples agree on the same value of the key, then they must be the same tuples. 

If some information is missing or is not known SQL uses the NULL. To check if a value is NULL we cannot say value=NULL since it's unknown we need to use ``IS NULL'' command of SQL. SQL offers the possibility to constrain a column to be NOT NULL or supports other constraints such as maximum number of values per attributes.
It's thanks to the schema and constraints that the databases understand the semantics of the data.

\section{Constraints in SQL}
A constraint is a relationship among data elements that the DBMS is required to enforce. The triggers are executed when a specified condition occurs, it's usually easier to implement than a constraint. 
Kinds of Constraints are:
\begin{itemize}
	\item (Foreign) Keys
	\item Value-Based constraint (constrain a certain value)
	\item Tuple-Based constraint (relationship among components)
	\item Assertion (any SQL boolean expression)
\end{itemize}

When creating a table we create a key using a constraint of UNIQUE or PRIMARY KEY. While expressing a Foreign Key we need to use the keyword REFERENCES after an attribute or as an element of the schema and the referenced attribute must be declared either as a PRIMARY KEY or UNIQUE. Here is an example:
\begin{lstlisting}
--For the single attribute key
CREATE TABLE table_1(
	attribute_1 CHAR(20) UNIQUE,
	attribute_2 VARCHAR(20)
);

--For the multi-attribute key
CREATE TABLE table_2(
	attribute_1 CHAR(20),
	attribute_2 VARCHAR(20),
	attribute_3 REAL,
	PRIMARY KEY (attribute_1,attribute_2)
);

--Foreign key reference example
CREATE TABLE table_1(
attribute_1 CHAR(20) UNIQUE,
attribute_2 VARCHAR(20)
);
CREATE TABLE table_2(
attribute_3 CHAR(20),
attribute_4 VARCHAR(20) REFERENCES table_1(attribute_1),
attribute_5 REAL,
);

--Foreign key as schema element
CREATE TABLE table_1(
attribute_1 CHAR(20) UNIQUE,
attribute_2 VARCHAR(20)
);
CREATE TABLE table_2(
attribute_3 CHAR(20),
attribute_4 VARCHAR(20),
attribute_5 REAL,
FOREIGN KEY(attribute_4) REFERENCES table_1(attribute_1)
);
\end{lstlisting}

Once we have a foreign key constraint there are two possible violations:
\begin{enumerate}
	\item Insert or update to table\_2 that introduces values not found in table\_1.
	\item Delete or update to table\_1 causing some tuples of table\_2 to be disrupted.
\end{enumerate}
In case of 1. we reject such operation. In case 2. in case of deletion or update we can proceed in three different ways:
\begin{enumerate}
	\item Default: Reject the modification like in case of 1.
	\item Cascade: Make the same changes in table\_2, in case of deletion (update) of a row of table\_1, delete (update) all the related tuples of table\_2.
	\item Set NULL: Change the element to NULL.
\end{enumerate}
One of these policies need to be chosen. We choose them while declaring the Foreign key and are of form (after foreign key declaration): ON [UPDATE,DELETE] [SET NULL CASCADE], if not declared the default(reject) is used. Here is an example:
\begin{lstlisting}
CREATE TABLE table_1(
attr_1 CHAR(20) UNIQUE,
attr_2 VARCHAR(20)
);
CREATE TABLE table_2(
attr_3 CHAR(20),
attr_4 VARCHAR(20),
attr_5 REAL,
FOREIGN KEY(attr_4)
	REFERENCES table_1(attr_1)
	ON DELETE SET NULL
	ON UPDATE CASCADE
);
\end{lstlisting}

Some other constraint are Attribute-Based checks to constraint on the value of a particular attribute in this case add CHECK($<$condition$>$) to the declaration for the attribute. We might use the name of the attribute in the condition but any other relation or attribute name must be in a subquery.
The checks are performed only when a value for that attribute is inserted or updated.
It can be added as a relation-schema element(like for the foreign key).

The last one is the Assertion, these are database-scheme elements and defined by: \newline CREATE ASSERTION $<$name$>$ CHECK ($<$condition$>$); the condition may refer to any relation or attribute in the database schema.
For example if there are two table Bars and Drinkers we want to assert that there cannot be more bars than drinkers the code is:
\begin{lstlisting}
	CREATE ASSERTION FewBar CHECK (
	(SELECT COUNT(*) FROM Bars) <=
	(SELECT COUNT(*) FROM DRINKERS)
	);
\end{lstlisting}
In principle, we must check the assertion after every modification to any relation of the database but a clever system can observe that only certain changes could cause a given assertion to be violated. 
Problem is the DBMS often can't tell when they need to be checked. 
The Triggers let the user decide when to check for any condition.
An example is when using a foreign key constraint, instead of rejecting insertions with unknown elements, a trigger can add that new element to the original table with other attributes set as NULL.
\end{document}
