\documentclass[11pt, twocolumn]{article}

\usepackage[english]{babel}
%\usepackage[italian]{babel}
\usepackage{amsmath}
\usepackage{amssymb}
\usepackage{hyperref}
\usepackage{listings}
\usepackage[margin=1in]{geometry}
\usepackage[utf8]{inputenc}
\counterwithin*{section}{part}

\title{\textbf{Decision Models}}
\author{}
\date{}

% contatti
% U14/2048, venerdì 16:30 - 17:30
% enza.messina@unimib.it
%
% libri
% _The Analystic Edge_, Bertsimas, O'Hair, Pulleyblank, Dynamic Idea LCC, 2016
% _Spreadsheet Modeling and decision analysis_, Cliff T. Radsgale, Thomson, 2007

\begin{document}
\maketitle
\begin{abstract}
  The course will have 28 hours of lectures and 18 of exercises.
  Books are not necessary: scientific papers are available on the Moodle platform.

  The exam will be different for students who attend the course: there are five periodic assignments (14 points) done individually, a group (2 or 3 people) project (16 points) and an optional oral exam (3 points).
  For students who do not attend the course there are just the group project and the oral exam.
  The project can also be done alone, but it is preferible to work in small groups; data can be taken from \url{kaggle.com}.
  The assignments will be evaluated 1 point each, and 3 of 5 will be discussed during the oral exam. \newline

  During the course there will be presented some advanced models to make decisions.
  \textit{Decision making} is the process of making a choice between a number of options.
  A \textit{decision model} is a computer-based system that make predictions based on data and statistics; a decision model may also predict what happen when an action is taken.
  A model is used to aid decisio-making, simulating elements and variables related to a particular decision.
  To make decisions, the process have to be optimized according to the data, usually using linear programming techniques.
  To \textit{analytics} is meant the discovery, interpretation and communication of patterns in the data, usually by a machine-learning program; can be descriptive, predictive or prescriptive.
  The prescriptive analysis is used in the ``what-if'' analysis and to determine what and how to do.

  The decision process concerns structured and familiar problems easy to solve with a clear goal, or programmed and repetitive tasks handeled with a routine approach.
\end{abstract}


\newpage
\part{Optimization}
Optimization is used to select the size of a neural network model, according to training data.
Often the optimization is done with different data (training-set) than data used by the model (test-set).

\end{document}
