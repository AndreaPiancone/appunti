\documentclass[a4page, 11pt]{article} %this is behaving like 
% sì, a4page è sottinteso

\usepackage[italian]{babel}
\usepackage{amsmath}
\usepackage{amssymb}
\usepackage{amsthm}
\usepackage{mathdots}
\usepackage{xfrac}
\usepackage{enumitem}
\usepackage[utf8]{inputenc}
\usepackage[margin=1.2in]{geometry}
\usepackage{cancel}
\usepackage{enumitem} %resuming enumerations
\usepackage{hyperref} %clickable TOC

\def\dep{\not\!\perp\!\!\!\perp}  % simbolo di dipendenza
\def\indep{\perp \!\!\! \perp }   % simbolo di indipendenza

\title{Statistical Modeling}
\author{}
\date{}


\begin{document}
\maketitle
\tableofcontents % magari risulta più facile da consultare per ripassi... @momo
\newpage

\section{Errori eteroschedastici}
Per ottenere stime efficienti per i parametri del modello lineare classico, gli errori devono essere \textit{omoschedastici}: la varianza degli errori $\varepsilon_i$ è costante $Var(\varepsilon_i) = E(\varepsilon_i^2) = \sigma^2$ e non dipende dal valore delle variabili indipendenti $E(\varepsilon_i) = E(\varepsilon_i | x_i)$.
Questo si verifica dal momento che il valore atteso del singolo errore è nulla $E(\varepsilon_i) = 0$.
Quando ciò non avviene, si è in presenza di errori \textit{eteroschedastici}: $Var(\varepsilon_i) = \sigma_i^2$; in tal caso il vettore $b$ dei parametri perde in efficienza\footnote{Si veda dimostrazione in appendice.} (non è più BLUE (\textit{Best Linear Unbiased Estimator}).
In forma matriciale, questo significa che si passa da una matrice diagonale con un valore costante a un'altra matrice diagonale i cui valori possono differire.

Inoltre, la stima campionaria $s^2$ di $\sigma^2$ tende a sottostimare il vero valore della distribuzione, dato che non si è più in presenza di una sola variabile casuale ma di molteplici.
Ne consegue che il test $T$ di Student ritorna valori erroneamente elevati, e quindi gli intervalli di confidenza risultano essere più stretti del reale, e i test di significatività sui parametri $b_j$ risultano più permissivi del dovuto.
Discorso analogo vale per i test basati sulla distribuzione $F$ di Snedcor.

Si può verificare se una distribuzione è o eteroschedastica (o omoschedastica) tramite diversi metodi grafici:
\begin{itemize}[noitemsep]
  \item scatter plot della variabile risposta vs esplicative $y \sim x_j$ (da effettuare per ogni $x_j$);
  \item scatter plot dei valori stimati vs residui $\hat{y} \sim \varepsilon$;
  \item scatter plot dei residui al quadrato vs i valori predetti $\varepsilon^2 \sim \hat{y}$;
  \item scatter plot dei valori osservati vs predetti $y \sim \hat{y}$;
  \item scatter plot dei residui vs variabili esplicative $\varepsilon \sim x_j$ (da effettuare per ogni $x_j$).
\end{itemize}
Esistono inoltre una serie di test statistici che offrono risultati numerici e meno interpretativi per verificare l'eteroschedasticità degli errori.

\paragraph{Testi di White. } Si basa sull’assunzione di omoschedasticità dei residui ($H_0: Var(\varepsilon_i) = \sigma^2$ contro $H_1: Var(\varepsilon_i) \neq \sigma^2$). % non è = \sigma_i perché è un'affermazione più forte: White l'ipotesi alternativa dice solamente che non è vera la nulla - @momo
%; viene perciò definita l’ipotesi nulla come $H_0:$ Var$(\varepsilon_i) = (\sigma^2)$ e l’ipotesi alternativa come $H_1:$ Var$(\varepsilon_i ) = ( \sigma^2_i )$.
Il test sfrutta la regressione $OLS$ del quadrato dei residui con le variabili esplicative, il loro quadrato e tutte le possibili interazioni $\varepsilon^2 \sim x_j, x_ix_j \hspace{5pt} \forall i, j < k$. %, i regressori al quadrato $x_j^2$ e le loro interazioni.
Attraverso l'indice di determinazione $R^2$ di tale regressione, ricavato dal rapporto tra la variabilità spiegata dalla regressione e la variabilità totale $R^2 = \frac{SSE}{TSS} = \frac{\sum(\hat{y}_i - y_i)^2}{\sum(y_i - \overline{y})^2}$, si calcola la statistica $LM = n R^2 \sim \chi^2_{k-1}$.
L’ipotesi nulla verrà rigettata se $LM$ risulterà maggiore del valore soglia della distribuzione $\chi^2$ (ovvero con p-value basso);
infatti se $R^2$ è maggiore di un certo valore soglia significa che le variabili esplicative sono realmente significative nello spiegare la variabilità dei residui.
\begin{equation*}
  LM = nR^2 = n\frac{\sum(\hat{y}_i - y_i)^2}{\sum(y_i - \overline{y})^2} \sim \chi^2_{k-1}
\end{equation*}

\paragraph{Test di Breusch-Pagan.} Anche in questo test, l’ipotesi nulla è quella di omoschedasticità ($H_0: Var(\varepsilon_i) = \sigma^2$).
Il test si basa sulla regressione di $\sfrac{\varepsilon^2_i}{s^2}$ (dove $s^2 = \frac{1}{n}\sum{\varepsilon_i^2}$) con le variabili esplicative.
Della regressione si calcola poi il coefficiente $R^2$ analogamente al test di White.
Essendo  $\varepsilon$ distribuita (sotto $H_0$) come una normale $N(0, \sigma^2)$, $\varepsilon^2 \sim \chi^2_{k-1}$; inoltre essendo già $s^2 \sim \chi^2_{n-k-1}$ (perché anch'esso somma di normali al quadrato) e $\varepsilon \indep X$ (sempre sotto $H_0$), $\sfrac{\varepsilon^2}{s^2}$ si distribuisce come un rapporto di $\chi^2$ indipendenti tra di loro, ovvero come una $F_{k-1, n-k-1}$ di Snedecor.
Si procede quindi effettuando un semplice test $F$ per l'accettazione di $H_0$.
Per risolvere tale problema si può procedere attraverso il metodo di stima $WLS$ (Weighted Least Squares).
\begin{equation*}
  BP = nR^2 \sim F_{k-1, n-k-1}
\end{equation*}

\section{Errori autocorrelati}
Per garantire stime efficienti del modello, bisogna verificare che gli errori $\varepsilon$ non siano tra di loro correlati, cioè non siano \textit{autocorrelati}.
Nel modello lineare classico si ipotizza infatti che:
\begin{equation*}
  Cov(\varepsilon_i, \varepsilon_j) = E(\varepsilon_i \varepsilon_j) = 0 \hspace{15pt} i \neq j
\end{equation*}
Tuttavia accade spesso, soprattutto in serie storiche o territoriali, che esista una correlazione tra errori in momenti successivi o territori vicini.
Gli errori correlati si possono scindere in due componenti:
$\varepsilon_{i.-1}^\#$ (errore ritardato di un tempo) e $\eta_i$ (errori omoschedastici IID, ovvero indipendentemente ed identicamente distribuiti in modo normale).
Una parte dell'errore è legata al suo valore ritardato, mentre l'altra è indipendente.
Possiamo classificare l’autocorrelazione in base al suo \textit{grado}: si dice autocorrelazione di primo grado quando gli errori sono correlati con il loro valore ritardato di un tempo; allo stesso modo si dice autocorrelazione di $i$-esimo grado quando gli errori sono correlati con il loro valore ritardato di $i$ gradi ($\varrho_{-i}$).
Gli errori autocorrelati non incidono sulle proprietà di linearità, correttezza e consistenza degli stimatori OLS (analogamente agli errori eteroschedastici), ma solo sull’efficienza (non sono più BLUE).
Come nel caso dell’eteroschedasticità, la stima della varianza dei parametri e relative inferenze non sono più corrette e affidabili (la statistica $T$ di Student ottiene valori erroneamente più elevati; gli intervalli di confidenza tendono ad essere più stretti e l'area di rifiuto del test anomalamente più ampia).

Per individuare la caratteristica di autocorrelazione esistono diversi sistemi grafici:
\begin{itemize}[noitemsep]
  \item scatter plot della variabile risposta vs esplicative $y \sim x_j$ (da effettuare per ogni $x_j$);
  \item scatter plot dei residui vs variabili esplicative $\varepsilon \sim x_j$ (da effettuare per ogni $x_j$);
  \item scatter plot dei residui vs ritardati $\varepsilon \sim \varepsilon_{-1}$;
  \item correlogramma: è un grafico imn cui sono mostrate le correlazioni a diversi gradi; analizzando l'$acf$ (la funzione di autocorrelazione dei residui) e $pacf$ si determina il tipo di modello autoregressivo.
\end{itemize}

Il test di Durbin-Watson invece offre uno strumento analitico per verificare la presenza di autocorrelazione a diversi gradi.
La sua ipotesi nulla è la mancanza di autocorrelazione: $H_0: \varrho_{-i} = 0$; mentre l'ipotesi alternativa può essere verificata su entrambe le code della distribuzione (bidirezionale) $H_1: \varrho_{-i} \neq 0$ o su una coda sola (unidirezionale destra o sinistra) $H_1: \varrho{_{-i}} ^>_< 0$.
La statistica $DW$ per l'autocorrelazione dei residui è definita come:
\begin{equation*}
  DW = \frac{\sum{(\varepsilon_i - \varepsilon_{i-1})^2}}{\sum{ \varepsilon_i^2}} \in [0, 4]
\end{equation*}
si dimostra inoltre che sotto ipotesi di omoschedasticità di $\varepsilon$, $DW = 2(1 - \varrho_{-i})$.
Il valore tende a $2$ in caso di mancanza di autocorrelazione, a $0$ in caso di massima autocorrelazione negativa ($DW = 0 \Leftrightarrow \varrho_{-i} = -1$) e $4$ positiva ($DW = 4 \Leftrightarrow \varrho_{-i} = +1$).
Convenzionalmente i valori critici per definire se l'autocorrelazione è significativa o meno sono $1$ e $3$ (è considerata significativa se $DW < 1 \lor DW > 3$).

Nel caso di autocorrelazione, il Teorema di Aitken stabilisce che nella classe degli stimatori lineari per il modello di regressione \textit{generalizzato} lo stimatore GLS è il più efficiente, ovvero è quello caratterizzato dalla minor varianza.

\section{Metodo di stima WLS, per soluzioni correlate, GLS }
\subsection*{Errori eteroschedastici e incorrelati: modello WLS}
Per errori eteroschedastici si intende quando la varianza dell’errore non è costante e il valore dipende dalle variabili esplicative ($\varepsilon \dep X$), violando quindi una delle ipotesi della regressione lineare classica.
Per tale motivo gli stimatori OLS non possono essere usati (in quanto non più efficienti); al contrario si possono utilizzare gli stimatori Weighted Least Squares (WLS) che permettono di stimare la varianza delle singole componenti erratiche $\varepsilon_i$ condizionatamente al vettore dei dati $x_i$.
Si esegue quindi una trasformazione della variabile risposta $y \rightarrow y^*$ e della matrice del disegno $X \rightarrow X^*$ per riportare la varianza degli errori ad una costante: si divide infatti ogni variabile per la radice di $h(i)$ (la varianza di $\varepsilon^*$, l'errore eteroschedastico).
La componente erratica dunque ha valore costante e il modello assume la forma:
\begin{equation*}
y^* = X^*\beta + \varepsilon
\end{equation*}
permettendo una stima $b$ dei parametri $\beta$ tramite il metodo OLS.

\subsection*{Errori omoschedastici e correlati: modello GLS}
Nel caso invece ci si trovi davanti ad errori autocorrelati come accade in serie storiche e territoriali è ragionevole ipotizzare che esista correlazione fra errori in momenti successivi o territori vicini.
Si parla di autocorrelazione se al variare di $X$ c'è fluttuazione dei valori di $Y$ con lo stesso segno (autocorrelazione \textit{positiva}), o segno opposto (autocorrelazione \textit{negativa}), oltre un certo intervallo di confidenza.
Si possono ricavare stime per errori correlati in modo più semplice tramite una stima dei parametri in una equazione che tenga conto della struttura di autocorrelazione seriale (metodo proposto da Durbin).
Bisogna innanzitutto stimare il coefficiente di autocorrelazione di $i$-esimo ordine attraverso un modello avente come variabile risposta gli errori $\varepsilon_t$ e come esplicative quelle già considerate più l’errore ritardato di $i$ tempi $\varepsilon_{t-i}$ e procedere alla stima del coefficiente di correlazione $\varrho$.
Vale infatti:
\begin{equation*}
  \varepsilon_t = \varrho \varepsilon_{t-i} - \delta_t
\end{equation*}
dove $\delta$ è la componente erratica che segue le ipotesi classiche.
Si moltiplica dunque ogni elemento dell'equazione ritardata per $\varrho$.
\begin{equation*}
  \varrho y_{t-i} = \varrho X_{t-i} \beta + \varrho \varepsilon_{t-i}
\end{equation*}
Infine si procede a sottrarre l'equazione ritardata moltiplicata per $\varrho$ all'equazione nella forma normale $y_t - \varrho y_{t-i}$ ottenendo un modello OLS per i parametri trasformati: 
\begin{alignat*}{6}
  &y_t^\# &&= &&X_t^\# &&\beta + &&\varepsilon_t^\# \\
  (y_t &- \varrho y_{t-i}) &&= (X_t &&- \varrho X_{t-i}) &&\beta + (\varepsilon_t &&- \varrho \varepsilon_{t-i})
\end{alignat*} 
Il modello rispetta tutte le proprietà classiche di correttezza, consistenza ed efficienza. 

In alternativa è possibile utilizzare un modello \textit{autoregressivo} che inserisce nell’equazione iniziale un errore ritardato che tenga conto dell’autocorrelazione di $i$-esimo ordine:
\begin{equation*}
  y = X \beta + AR_i + \varepsilon
\end{equation*}
con 
\begin{align*}
  AR_i + \varepsilon &= v \\
  Corr(v_j,v_k) &= 0 \hspace{15pt} j \neq k
\end{align*}


\subsection*{Errori eteroschedastici e correlati: stimatore GLS}
Nel caso in cui gli errori non siano sferici in quanto eteroschedastici e correlati si utilizzano gli stimatori dei minimi quadrati generalizzati ($GLS$) interpretabili in modo analogo al modello classico in quanto stimatori $OLS$ basati su variabili trasformate per mezzo delle proprietà degli autovettori e autovalori ricavati dalla matrice dei residui $\Sigma_\varepsilon$.
Nello specifico si procede ad effettuare una \textit{decomposizione spettrale} della matrice degli errori:
\begin{equation*}
  \Sigma_\varepsilon = \sigma^2 V V^\prime
\end{equation*}
con
\begin{equation*}
V = \sigma (\sqrt{AL})A^\prime
\end{equation*}
dove $A$ è la \textit{matrice degli autovettori} e $L$ è la matrice diagonale degli autovalori di $\Sigma_\varepsilon$.
A questo punto premoltiplicando per $V^{-1}$ le componenti del modello si ottiene una nuova funzione con variabili trasformate e $\Sigma_{\varepsilon^\circ}$ omoschedastica ed incorrelata:
\begin{align*}
  V^{-1}y &= y^\circ \\
  V^{-1}X &= X^\circ \\
  y^\circ &= X^\circ \beta^\circ + \varepsilon
\end{align*}
Lo stimatore $b^\circ$ risulta godere delle proprietà di correttezza e consistenza; inoltre secondo il Teorema di Aitken è lo stimatore più efficiente per il modello \textit{generalizzato} (nonostante abbia una varianza maggiore rispetto al metodo OLS $\sigma^2 (X^{\circ \prime} X^\circ)^{-1} > \sigma^2 (X^\prime X)^{-1}$).
Tuttavia la stima GLS necessita di assumere come nota la matrice di varianze e covarianze dei residui $\Sigma_\varepsilon$, o almeno poter calcolare una sua stima, a patto però che sia consistente al limite:
\begin{equation*}
  \lim_{n\to \infty} S_\varepsilon = \Sigma_\varepsilon
\end{equation*}
A questo punto si possono utilizzare gli stimatori FGLS (\textit{Feasible Generalized Least Squares}).
Spesso la soluzione di applicare i FGLS viene intrapresa anche in caso di semplice eteroschedasticità o semplice autocorrelazione, o sospette tali, poiché vige il principio di precauzione.

\section{Multicollinearità }
Se la matrice $(X^\prime X)$ non è invertibile o ha determinante prossimo allo 0, le stime non possono essere calcolate (coefficienti sotto identificati, poiché non si dispone di sufficiente informazione per stimarli) o non sono affidabili (coefficienti empiricamente sotto identificati).

Tale problema si verifica quando almeno una delle variabili è correlata linearmente alle altre: si ha quindi multicollinearità.
In questo caso la matrice $(X^\prime X)$ è detta singolare e l'inversa $(X^\prime X)^{-1}$ non è unica. 

Esistono due tipi di collinearità:
\begin{itemize}[noitemsep]
  \item perfetta: sussiste quando almeno una variabile esplicativa è una combinazione lineare perfetta delle altre: questa viola le proprietà del modello lineare classico; solitamente ciò si verifica per un errore nella definizione dei regressori o per la presenza di due variabili direttamente dipendenti una dall'altra (ad esempio \textit{titolo di studio} e \textit{anni di studio});
  \item imperfetta: sussiste in caso di forte correlazione tra i regressori e dunque il determinante della matrice dei coefficienti tende a $0$; ciò non rende impossibilità la stima dei coefficienti ma da origine a stime fortemente distorte e caratterizzate da un'alta varianza.
\end{itemize}

L'errore di stima provoca un aumento della varianza dello stimatore dei coefficienti $b$, da cui deriva una sovrastima delle dimensioni degli intervalli di confidenza (che risultano più ampi del dovuto) e una maggiore zona di accettazione nei test statistici di quanto sarebbe corretto.
Inoltre, l'aggiunta di una variabile fortemente correlata ad una già presente nel modello aggiunge poca informazione (sarebbe opportuno calcolare la correlazione \textit{spuria}).

Per individuare fenomeni di multicollinearità, è buona norma, in prima istanza, generare una \textit{matrice di correlazione} tra tutte le variabili così da identificare rapidamente possibili variabili collineari.
Auspicabilmente infatti è preferibile avere forte correlazione tra $y$ e i regressori $x_j$, e bassa correlazione tra questi ultimi.
Esistono inoltre metodi analitici:

\paragraph{Indice di tolleranza.} Misura il grado di interrelazione di una variabile indipendente rispetto alle altre.
Si calcola come $TOL = 1 - R^2 \in [0, 1]$ dove $R^2$ è il coefficiente di correlazione della regressione di una variabile esplicativa $x_j$ (usata come risposta) in funzione delle altre.
Valori alti indicano una bassa multicollinearità tra la singola variabile $x_j$ e le altre.

\paragraph{Varianza multifattoriale (o VIF).} È il reciproco della tolleranza $VIF = TOL^{-1}$.
Valori di tale indice variano tra $0$ e $\infty$, ma si considera significativo già se superiore a $20$ ($TOL = 0.05$) indicando uno stretto rapporto tra la variabile considerata e le altre del modello, ovvero un eccessivo grado di \textit{multicollinearità}.
Vanno considerate con attenzione anche quelle variabili con valori di VIF maggiori di $10$ ($TOL = 0.1$);

\paragraph{L'indice di condizione.} È dato dalla radice del rapporto tra l’autovalore massimo della matrice $(X^\prime X)$ e gli altri autovalori.
Quando risulta essere maggiore di $30$ si considera significativa la presenza di \textit{multicollinearità}.
Tale convinzione viene rafforzata se un autovalore con \textit{condition index} maggiore di $30$ contribuisce a spiegare elevate quote di varianza di due o più variabili.

\section{Linearità}
La relazione ipotizzata tra la nostra variabile dipendente $y$ e le singole variabili esplicative $x$ è di tipo: $y = f(x)$, con $f$ lineare.

L'approssimazione lineare non è sempre la migliore.
Per validare la presenza di ciascun regressore all’interno dei diversi modelli dobbiamo quindi verificare la linearità di tale relazione.
Dunque, la variabile risposta deve essere una combinazione lineare di variabili esplicative e di parametri lineari.

Se una relazione tra $y$ e $X$ non è lineare, allora l’effetto su $y$ ($\partial y$)  di una variazione in $X$ ($\partial X$) dipende puntualmente dal valore di $X$ poiché l’effetto marginale di $X$ non è costante. 

In questo caso, una regressione lineare è mal specificata: la forma funzionale è errata e lo stimatore dell’effetto su $y$ di $X$ non è corretto nemmeno sulla media. 
Può capitare ad esempio che l'indice $R^{2}$ sia elevato ma che non ci sia linearità perchè c'è sia una componente lineare sia una non lineare. 

Per verificare la presenza (o meno) di linearità è possibile ricorrere ad alcuni grafici: 
\begin{itemize}[noitemsep]
\item scatter plot della variabile risposta vs esplicative $y \sim x_j$ (da effettuare per ogni $x_j$);
\item scatter plot dei residui vs la variabile risposta $\varepsilon \sim y$ (non deve presentare andamenti sistematici);
  \item scatter plot dei residui vs valori previsti $\varepsilon \sim \hat{y}$ (l'andamento deve essere regolare).
\end{itemize}
  
È da notare che la non linearità potrebbe dipendere anche solo da una o da alcune variabili esplicative e non necessariamente da tutte.
Quando è presente non linearità dei parametri, potrebbe esistere una trasformazione che li renda lineari (caso linearizzabile) oppure che questi siano espressi in una forma intrinsecamente non lineare.

Nel primo caso si procede innanzitutto alla linearizzazione del parametro (o della variabile) \textit{non lineare} con una trasformazione che lo renda \textit{lineare}, poi si procede alla stima OLS ed infine si applica la trasformazione inversa ricavando la stima del parametro originale. 
Nel caso invece di componenti intrinsecamente non lineari si procede allora alla stima attraverso gli stimatori NLS (minimi quadrati non lineari) che sfruttano algoritmi numerici nei software per affrontare il problema di minimizzazione non lineare.

Volendo utilizzare funzioni di variabili indipendenti non lineari in X possiamo riformulare una vasta famiglia di funzioni di regressione lineare come regressioni multiple.

Tra le funzioni non lineari le più utilizzate sono le polinomiali e le trasformazioni logaritmiche. 

Tra la trasformazioni logaritmiche esistono tre modelli principali: 
\begin{itemize}[noitemsep]
\item Linear-Log, in cui ad un incremento percentuale della variabile indipendente corrisponde un incremento nominale lineare della variabile dipendente;
\item Log-Linear, in cui ad un incremento nominale dell’esplicativa corrisponde un incremento percentuale della risposta;
\item Log-Log, in cui entrambi gli incrementi sono percentuali.
\end{itemize}
Tuttavia la trasformazione di una variabile, eccettuando casi particolari in cui il dominio lo permette (la trasformata $log$-$log$ in un grafico quantità-prezzo indica l'\textit{elasticità}) , rende di difficile interpretazione il modello.

\section{Non normalità}

Quando gli errori $\varepsilon_i$ sono indipendenti e identicamente distribuiti come $N(0,\sigma^{2})$ si possono ricavare la distribuzione degli stimatori, i test statistici, gli intervalli di confidenza e le proprietà ottimali (inoltre stima di massima verosimiglianza $ML$ coincide con stima dei minimi quadrati $OLS$). 
Nel caso in cui gli errori non siano normali, se tuttavia i campioni sono sufficientemente larghi per il Teorema del limite centrale, la distribuzione degli errori tende \textit{asintoticamente} alla normalità. Se ciò non accade non è possibile applicare test e intervalli di confidenza perchè essi sono basati tutti sull’ipotesi di normalità degli errori.
%till here @Dario

Conseguenze della violazione della normalità:
\begin{enumerate}[noitemsep]
\item I parametri $\beta$ possono essere espressi come combinazione lineare degli errori, per cui se gli errori non sono normali anch’essi non sono più normali;
\item Non è più possibile ricavare test basati sulla normale standardizzata;
\item Non è più possibile ricavare intervalli di confidenza per i parametri basati sulla normale standardizzata;
\item Le stime OLS non coincidono con le stime ML ottenute attraverso il metodo della massima verosimiglianza, quindi non sono i più efficienti tra tutti gli stimatori corretti (non sono più VUE), ma rimangono corretti e consistenti, e i più efficienti tra tutti gli stimatori lineari (sono ancora BLUE).
\end{enumerate}
Non coincidendo più le stime OLS ed ML, alcuni software statistici potrebbero fornire stime non attendibili.

Per individuare casi di non normalità è opportuno:
\begin{itemize}[noitemsep]
\item Osservare indici descrittivi;
\item Effettuare rappresentazioni grafiche;
\item Effettuare test non parametrici (ovvero realizzati con lo scopo di testare la distribuzione del parametro sotto osservazione).
\end{itemize}
Tra gli indici descrittivi possiamo, in prima istanza, osservare indicatori quali moda, media e mediana dei residui $\varepsilon$: quando questi coincidono, si può supporre che la distribuzione sia normale; questa operazione può essere effettuata rapidamente utilizzando un box-plot.
Si possono usare comunque altre distribuzioni grafiche, quali:
\begin{itemize}[noitemsep]
  \item istogramma della distribuzione dei residui $\varepsilon$;
  \item plot della distribuzione cumulata dei residui, alla ricerca di evidenti irregolarità;
  \item qq-plot (quantile vs quantile) della distribuzione dei residui $\varepsilon$ vs la normale standard (i punti si dovrebbero disporre sulla bisettrice);
  \item pp-plot (probability vs probability) della distribuzione cumulata dei residui $\varepsilon$ vs la cumulata della normale standard. 
\end{itemize}
Esistono, infine, alcuni test non parametrici che non si basano su ipotesi sulla distribuzione ma che sono appunto detti non parametrici poichè testano la distribuzione dei parametri. Per questo motivo sono molto utili per analizzare problemi di normalità dei residui.

\paragraph{Test di Shapiro-Wilk.} L'ipotesi nulla $H_0 : \varepsilon \sim N(0, \sigma^2)$ è accettata con valori alti dell'indice $W$; tuttavia essendo caratterizzato da una forte asimmetria può portare a un rifiuto dell'ipotesi nulla anche in presenza di distribuzione normale.
\begin{equation*}
W = \frac{\sum(a_i \varepsilon_{(i)})^2}{\sum\varepsilon_i^2} \in [0, 1]
\end{equation*}

\paragraph{Test di Kolmogorov Smirnov.} Anche qui $H_0 : \varepsilon \sim N(0, \sigma^2)$, ma si basa sul calcolo della statistica test $D$, calcolato come somma in valore assoluto della differenza tra le frequenze cumulate della distribuzione empirica da testare e quelle della normale, una volta definite delle classi di eguale ampiezza.
$D$ viene poi messa a confronto con le apposite tavole (essendo una statistica tabulata), ed in caso di superamento del valore critico in base al livello di significatività scelto comporterà il rifiuto o l'accettazione di $H_0$.

\paragraph{Skewness test.} È un test di asimmetria ($H_0 : \varepsilon : P(\varepsilon_i < 0) = P(\varepsilon_i > 0) \forall i$) ma rigettando $H_0$ si rigetta l'ipotesi di normalità della distribuzione; l'accettazione dell'ipotesi nulla tuttavia non fornisce indicazioni sulla reale distribuzione.
\begin{equation*}
  S = \frac{\frac{1}{n}\sum\varepsilon_i^3}{\sigma^3} \in (-\infty, +\infty)
\end{equation*}
La distribuzione è considerata simmetrica con $-1 < S < +1$.
\paragraph{Kurtosis test.} Simile a quello per l'asimmetria, fornisce solamente informazioni sulla curtosi della distribuzione:
\begin{equation*}
  K = \frac{\frac{1}{n}\sum\varepsilon_i^4}{(\frac{1}{n}\sum\varepsilon_i^2)^2} = \frac{\frac{1}{n}\sum\varepsilon_i^4}{\sigma^4} \in (-\infty, +\infty)
\end{equation*}
La distribuzione normale ha curtosi pari a $3$, quindi il test è considerato significativo per valori prossimi. \newline

I problemi di non normalità possono essere risolti usando una trasformata della variabile dipendente $y$.
La trasformazione può migliorare la relazione lineare tra la variabile dipendente e le variabili indipendenti.
Le trasformazioni più diffuse sono:
\begin{itemize}[noitemsep]
  \item $\log(y)$: quando $\sigma^2_\varepsilon \dep y$ o la distribuzione dell'errore ha asimmetria \textit{positiva};
  \item $y^2$ quando $\sigma^2_\varepsilon$ è proporzionale a $\overline{y}$ o quando la distribuzione dell'errore ha asimmetria \textit{negativa};
  \item $\sqrt{Y}$ quando $\sigma^2_\varepsilon$ è proporzionale a $\overline{y}$;
  \item $Y^{-1}$ quando $\sigma^2_\varepsilon$ cresce significativamente al crescere di $y$.
\end{itemize}

\section{Outlier}
I valori cosiddetti \textit{outlier} possono essere distinti in valori anomali (che si discostano in modo rilevante dall’andamento generale) e punti influenti (che influenzano in misura rilevante le stime).
Un'osservazione può appartenere a entrambe le categorie o a una sola in modo del tutto casuale.

Gli outliers possono essere identificati visivamente tramite rappresentazioni grafiche (box-plot e scatter-plot) su distribuzioni a due dimensioni; all'aumentare del numero di dimensioni si perde la possibilità di individuarli in modo grafico e sono necessari indici numerici.

\paragraph{\textit{Leverage values}.} Definendo la matrice di proiezione $H = X(X^{\prime}X)^{-1}X^\prime$, si considerano gli elementi $h_{i.i}$ sulla diagonale principale (chiamati \textit{leverage}) che rappresentano l'impatto dell'osservazione $i$-esima sulla capacità del modello di predire tutti i casi.

Si dimostra che il valor medio del leverage è:
\begin{equation*} 
\overline{h} = \frac{k - 1}{n}
\end{equation*}
con $k$ che rappresenta il numero di variabili esplicative e $n$ il numero di osservazioni.
Un valore \textit{leverage} si considera significativamente alto se supera $2$ o $3$ volte (scelto in modo arbitrario) il suo valore medio:
\begin{equation*}
h_{i.i} > 2\frac{k - 1}{n}
\end{equation*}

\paragraph{Residui standardizzati.} I residui $\varepsilon$ possono essere calcolati come:
\begin{equation*}
\varepsilon = (I-H)y
\end{equation*}
È dunque possibile scrivere la loro varianza come:
\begin{equation*}
Var(\varepsilon_i) = (1 - h_{i.i})\sigma^2
\end{equation*}
Si possono quindi calcolare i residui \textit{standardizzati} grazie alla formula:
\begin{equation*}
\varepsilon_i^* = \frac{\varepsilon_i}{\sigma \sqrt{(1-h_{i.i})}} \sim N(0, 1)
\end{equation*}
Si usano quindi le tavole della normale per verificare la presenza di \textit{outliers}: il $95\%$ della popolazione assume valori compresi tra $-2 < \varepsilon_i^* < +2$, e così via.
Si considera \textit{outlier} un valore il cui valore assoluto $|\varepsilon_i^*| > 3$ (dato che la probabilità che si verifichi casualmente è particolarmente bassa).

\paragraph{Residui studentizzati.} Concettualmente identico ai residui standardizzati, però con varianza campionaria $s^2$ (formule e test sono i medesimi).
La formula è la medesima ma con lo stimatore $s^2$ al posto del valore reale $\sigma^2$.

\paragraph{Residui \textit{jack-knife}.} Usati con campioni di numerosità non elevata, nei residui \textit{jack-knife} la varianza $s^2_i$ è calcolata eliminando momentaneamente la $i$-esima variabile dal modello e stimando nuovamente i parametri.

\paragraph{Covrati.} Misura la variazione nel determinante della matrice delle covarianze delle stime eliminando la $i$-esima osservazione.
Eliminando infatti il valore $i$-esimo si provoca una variazione nel determinante che vado a quantificare.
\begin{equation*}
\text{COVRATIO}_i = \frac{det(\Sigma_i)}{det(\Sigma)} = \frac{det(\frac{1}{n - 1}\tilde{X}^{\prime}_i\tilde{X}_i)}{\det(\frac{1}{n}\tilde{X}^{\prime}\tilde{X})}
\end{equation*}
Si considera significativo se supera la soglia:
\begin{equation*}
1 \pm 3 \  \sqrt{\frac{k + 1}{n}}
\end{equation*}

\paragraph{Dfitts.} Misura l’influenza dell'$i$-esima osservazione sulla stima dei coefficienti di regressione e sulla loro varianza, calcolandolo sul valore stimato della variabile risposta $y$.
\begin{equation*}
\text{DFITTS}_i = \frac{\hat{y} - \hat{y}_i}{s_i\sqrt{h_{i.i}}}
\end{equation*}
Si considera significativo se supera il valore soglia:
\begin{equation*}
\pm 2 \sqrt{\frac{k+1}{n}}
\end{equation*}

\paragraph{Dfbetas.} Misura l’influenza dell'$i$-esima osservazione sulle stime dei coefficienti di regressione (considerandoli singolarmente).
Valori elevati indicano che l'osservazione influisce molto sulla stima dei parametri.
\begin{equation*}
\text{DFBETAS}_i = \beta - \beta_i = X_i(X^{\prime}X)^{-1}\frac{\varepsilon_i}{1-h_{ii}}
\end{equation*}
Si considera significativo se è superato il valore $2$ (o $2 \sqrt{n}$) per almeno un parametro $\beta$: risulta essere un indice più stringente dei precedenti.

\paragraph{Distanza di Cook.} Misura l’influenza dell'$i$-esima osservazione sulla stima dei coefficienti di regressione nel loro complesso.
\begin{equation*}
D_i = \frac{\sum(\hat{y}_j - \hat{y}_{j.i})^2}{k s^2} \sim F_{(k, n-k)}
\end{equation*}
Si considerano significative le distanze che superano il valore soglia $1$ o $^4/_n$.


\section{Modello lineare classico multivariato}
Si intende con modello lineare classico \textit{multivariato} l'estensione del modello classico multiplo a $m$ variabili risposta $y_j$; ogni variabile risposta è legata alle stesse variabili esplicative.

Per l'$i$-esimo individuo si ha:
\begin{align*}
y_i &= [y_{i.1} \hspace{5pt} \dots \hspace{5pt} y_{i.j} \hspace{5pt} \dots \hspace{5pt} y_{i.m}]\\
z_i &= [1, z_{i.1} \hspace{5pt} \dots \hspace{5pt} z_{i.k} \hspace{5pt} \dots \hspace{5pt} z_{i.r}]\\
\varepsilon_i &= [\varepsilon_{i.1} \hspace{5pt} \dots \hspace{5pt} \varepsilon_{i.j} \hspace{5pt} \dots \hspace{5pt} \varepsilon_{i.m}]
\end{align*}

Mentre la matrice dei parametri $\beta$, di dimensioni $m \times r+1$, diventa:
\begin{equation*}
\beta =
\begin{bmatrix}
   \beta_{1.0} & \cdots & \beta_{1.k} & \cdots & \beta_{1.r} \\
   \vdots      & \ddots & \vdots      & \iddots & \vdots \\
   \beta_{j.0} & \cdots & \beta_{j.k} & \cdots & \beta_{j.r} \\
   \vdots      & \iddots& \vdots      & \ddots & \vdots \\
   \beta_{m.0} & \cdots & \beta_{m.k} & \cdots & \beta_{m.r} 
\end{bmatrix}
\end{equation*}
In sintesi ogni \textit{riga} della matrice dei parametri $\beta$ si riferisce ad una variabile risposta $y_{1 \dots m}$ mentre ogni colonna si riferisce ad una variabile esplicativa $z_{1 \dots r}$. 
\newline
Nel suo complesso perciò il modello multivariato appare come:
\begin{equation*}
  Y_{m \times n} = B_{m \times r + 1} Z_{r+1 \times n} + E_{m \times n}
\end{equation*}
o altrimenti scritto come:
\begin{equation*}
  y_j = \beta_j Z + \varepsilon_j \hspace{15pt} j = 1 \dots m
\end{equation*}
Ogni colonna della matrice $Y$ rappresenta un carattere, mentre ogni riga rappresenta i valori dei caratteri per un singolo individuo.

Le ipotesi del modello sono analoghe a quelle formulate per il modello univariato ma, essendo applicate su più variabili dipendenti, risultano molto più stringenti.
\begin{itemize}[noitemsep]
\item I parametri sono lineari.
\item I valori attesi degli errori casuali sono nulli: $E(\varepsilon_{i.j}) = 0$.
\item Gli errori casuali all’interno gli ogni equazione e \textit{anche tra diverse equazioni} sono omoschedastici e incorrelati.
  La matrice di varianze e covarianze dei residui assume infatti la forma:
\begin{equation*}
  \Sigma_E =
  \begin{bmatrix}
    \sigma^2 I_n   & \cdots  & 0_{n \times n} & \cdots  & 0_{n \times n} \\
    \vdots         & \ddots  & \vdots         & \iddots & \vdots \\
    0_{n \times n} & \cdots  & \sigma^2 I_n   & \cdots  & 0_{n \times n} \\
    \vdots         & \iddots & \vdots         & \ddots  & \vdots \\
    0_{n \times n} & \cdots  & 0_{n \times n} & \cdots  & \sigma^2 I_n 
  \end{bmatrix}
\end{equation*}
Con dimensione $nm \times nm$ poiché ognuna delle $m$ matrici $\Sigma_{\varepsilon_j}$ relativa ad ogni singola equazione $j$ è di dimensione $n \times n$; mentre le matrici sulla diagonale principale rappresentano gli errori relativi alla medesima equazione, le matrici $0$ racchiudono le correlazioni fra gli errori relativi ad equazioni diverse.
Infatti le variabili risposta si presume siano indipendenti le une dalle altre sia per lo stesso individuo ($y_{i.j} \indep y_{i.k} \hspace{5pt} \forall j, k : k \ne i$, sulla diagonale della matrice) che per individui diversi; eventuali correlazioni sono da considerarsi \textit{spurie} data la correlazione alle stesse variabili esplicative.
\item Le  variabili esplicative sono non stocastiche: per ogni osservazione $i$, i valori $z_{i.k}$ sono costanti, mentre il corrispondente valore di ogni $y_{i.j}$ è una variabile casuale influenzata dagli errori casuali.
\item La matrice $Z$ ha rango pieno $rk(Z) = r + 1$, ciò vuol dire che nessuna variabile esplicativa è una combinazione lineare delle altre; contrariamente la matrice $Z^{\prime}Z$ non sarebbe invertibile e non si potrebbe calcolare lo stimatore dei minimi quadrati.
\item La numerosità della popolazione $n$ è maggiore del numero dei parametri stimati più l'intercetta $n > r + 1$, perciò per ogni equazione le stime dei minimi quadrati di $\hat{\beta}$ sono trovate in modo analogo al caso univariato:
\begin{equation*}
  \hat{\beta} = (Z^{\prime}Z)^{-1} Z^{\prime} y_j
\end{equation*}
Di conseguenza, nel modello multivariato classico calcolare le soluzioni per ogni variabile dipendente $y$ singolarmente o in gruppo è identico.
\item Gli errori $E$ si distribuiscono come una normale multivariata:
\begin{equation*}
E \sim N(0,s^2 I_{nm})
\end{equation*}
Permane la condizione di ortogonalità poiché i residui sono incorrelati sia con le variabili esplicative $Z$ che con i valori predetti della variabile dipendente $\hat{Y}$.
\end{itemize}

Inoltre poiché $Y = \hat{Y} + \hat{E}$, si ha che:
\begin{alignat*}{5}
  Y&Y^{\prime} &&= \hat{B}Z&&Z^{\prime}\hat{B^\prime} &&+  \hat{E}&&\hat{E^\prime} \\
  &\Sigma_Y &&= &&\hat{H} &&+ &&\hat{\Sigma_E}
\end{alignat*}
Con $\Sigma_Y$ matrice di varianze e covariante di $Y$, $\hat{H}$ matrice di varianze e covariante spiegate e $\Sigma_E$ matrice di varianze e covariante residue.

La grossa differenza tra la soluzione univariata e multivariata, però, sta nelle \textit{covarianze}, poiché varianze spiegate e residue non sono scalari ma matrici; occorre quindi tenere conto delle correlazioni tra le soluzioni.

Nel caso multivariato classico infatti le parti diagonali di $ \Sigma_E$ e $\hat{H}$ sono \textit{identiche}.
\begin{alignat*}{5}
  Y&Y^{\prime} &&= \hat{B}Z&&Z^{\prime}\hat{B^\prime} &&+ \sigma^2I_{nm} \\
  &\Sigma_Y &&= &&\hat{H} &&+ \sigma^2I_{nm}
\end{alignat*}
L’$R^2$ complessivo del modello è una media degli $R^2$ delle singole equazioni pesata per la numerosità dei casi (che può mutare a causa dei valori nulli).


\section{Inferenza nella Regressione Multivariata}
Gli stimatori OLS sono corretti ed efficienti, poiché il Teorema di Gauss-Markov vale anche per il caso multivariato.
Infatti nell'ambito degli stimatori lineari e corretti del vettore dei parametri, lo stimatore $b$ dei minimi quadrati è quello a \textit{varianza minore}.
Inoltre, per il modello di regressione multivariata con rango pieno ed errori $E$ normalmente distribuiti anche le $m$ variabili dipendenti $Y$ sono distribuite secondo una normale multivariata come anche i parametri stimati $b$
\begin{align*}
  Y &\sim N(BZ, \Sigma_Y) \\
  b & \sim N(\beta, \hat{H})
\end{align*}
con $\hat{H}$ matrice di varianze e covarianze spiegate della popolazione, positiva definita ed efficiente, e che si dimostra essere distribuita in modo indipendente da $E$ matrice degli errori.
È però il caso di notare che $\Sigma_Y$ e $\hat{H}$ sono entrambe matrici di varianza e covarianza non diagonali e, di conseguenza, sono influenzate dalle correlazioni.

Si definisce invece \textit{varianza generalizzata} di $\hat{H}$ il suo determinante $det(\hat{H})$; per le distribuzioni multivariate si usa questo indice di variabilità in quanto ha il vantaggio di essere univariato e permette quindi di costruire facilmente test $t$ ed $F$.
La varianza generalizzata inoltre considera la correlazione delle variabili, e si  dimostra infatti uguale a $0$ in caso di presenza di:
\begin{itemize}[noitemsep]
\item variabile costante nelle unità statistiche;
\item variabile perfettamente correlata con un'altra (rango non pieno);
\item variabile combinazione lineare di altre variabili.
\end{itemize}
Analogamente si definisce \textit{varianza generalizzata} di $\Sigma_E$ il determinante della matrice di varianza-covarianza residua. \newline

Considerando che $\hat{H}$ si distribuisce come $\hat{H} \sim W_r$ (variabile casuale di Wishart, una generalizzazione multivariata della $\Gamma$) e $\Sigma_E \sim W_{r - n}$ (e inoltre $\hat{H} \indep \Sigma_E$), si può quindi definire il loro rapporto come test di verosimiglianza Lambda di Wilks:
\begin{equation*}
  \Lambda = \frac{det(\Sigma_E)}{det(\Sigma_E + \hat{H})}
\end{equation*}
Che si distribuisce \textit{asintoticamente} come una $\chi^2_{mr}$.
Sempre da $\Lambda$, inoltre, si ricava una distribuzione asintotica di $F$
\begin{equation*}
F = \frac{1 - \Lambda}{\Lambda}
\end{equation*}
che nel caso sia rispettata l'ipotesi di normalità dei residui
\begin{equation*}
E \sim N(0,\sigma^2 I_{nm})
\end{equation*}
permette di costruire \textit{test multivariati} per i parametri  del modello analoghi a quelli costruiti utilizzando $F$ nel caso univariato.
Il test del rapporto di verosimiglianza Lambda di Wilks assume come ipotesi nulla: 
\begin{equation*}
H_0 : \hat{B} = 0 
\end{equation*} 
per cui sotto $H_0$, $\Lambda \to 1$ dato che numeratore e denominatore tenderebbero a coincidere, mentre $F \to 0$.
Analogamente si possono costruire intervalli di confidenza per i parametri e per i valori predetti delle $Y$.

Esistono inoltre altri test che possiedono la stessa distribuzione, implicazioni e ipotesi $H_0$ della $\Lambda$ di Wilks:
\begin{itemize}[noitemsep]
\item Traccia di Lawney-Hotelling:
\begin{equation*}
LH = \frac{det(\hat{H})}{det(\Sigma_E)}
\end{equation*}
\item Traccia di Pillai:
\begin{equation*}
P = \frac{det(\hat{H})}{det(\hat{H} + \Sigma_E)}
\end{equation*}
\item Massimo autovalore di Roy:
\begin{equation*}
\max(\lambda_i) \hspace{15pt} \text{con $\lambda$ autovettore di} \frac{\hat{H}}{\hat{H} + \Sigma_E}
\end{equation*}
\end{itemize}

In modo analogo, si possono costruire altri test con altre ipotesi nulle $H_0$:
\begin{itemize}[noitemsep]
\item Test sulla non significatività di un gruppo di variabili esplicative rispetto a tutte le variabili dipendenti $H_0 : \hat{B} = 0$);
\item Test sull’uguaglianza dei parametri relativi a diversi gruppi di variabili esplicative nelle singole equazioni $H_0 : B_{kj} = B_{gj}$;
\item Test sull'uguaglianza dei parametri relativi alle stesse variabili in coppie di diverse equazioni $H_0 : B_{cA} = B_{vA}$
\end{itemize}


\section{Modello lineare generalizzato}
Il modello lineare multivariato generalizzato supera le ipotesi, molto stringenti, del modello lineare multivariato classico, rimuovendo le ipotesi sugli errori.
La formula diventa
\begin{equation*}
  Y = BZ+E
\end{equation*}
in cui la matrice di covarianza degli errori $E$ non è più necessariamente diagonale e gli errori possono essere eteroschedastici. 

Nell’ipotesi \textit{classica} si ipotizza che gli errori siano:
\begin{itemize}[noitemsep]
\item omoschedastici e incorrelati all’interno delle stesse equazioni: in ogni equazione, la varianza spiegata è uguale per ogni individuo e non si hanno correlazioni tra gli errori ($E_{i \cdot} = \sigma^2 I_m$);
\item omoschedastici e incorrelati tra equazioni diverse: discorso analogo per le diverse equazioni ($E_{\cdot j} = \sigma^2I_n$);
\end{itemize}

Con le ipotesi classiche, costruire un modello lineare generalizzato equivale a costruire $m$ modelli lineari, uno per variabile risposta, indipendenti tra di loro: la varianza residua (ovvero la componente erratica) non è influenzata dal risultato degli altri individui né delle altre variabili risposta.

Si ipotizza un modello intermedio in cui gli errori sono:
\begin{itemize}[noitemsep]
\item omoschedastici e incorrelati all’interno delle stesse equazioni, come nel modello classico;
\item eteroschedastici e correlati tra equazioni diverse: la varianza spiegata è unica per ogni individuo e il valore di una variabile dipendente può influenzare gli altri dello stesso individuo ($E_{\cdot.j} \ne \sigma^2 I_n$);
\end{itemize}
A differenza del modello precedente, il valore di una variabile risposta influenza le altre dello stesso individuo e si 

Facendo cadere le ultime ipotesi, si ottiene il modello lineare \textit{generalizzato}, in cui gli errori sono:
\begin{itemize}[noitemsep]
\item eteroschedastici e correlati all'interno delle stesse equazioni: il comportamento di un individuo può influenzare gli altri, e ogni individuo ha una varianza unica ($E_{i \cdot} \ne \sigma^2 I_m$);
  \item eteroschedastici e correlati tra equazioni diverse: come nel modello intermedio.
\end{itemize}
Si ha quindi l'opposto dell'ipotesi classica: ogni osservazione influenza le altre e ogni variabile dipendente è influenzata dalle altre.

Occorre quindi usare non le singoli sottomatrici di correlazione degli errori $\Sigma_{E_i}$, ma la matrice $\Sigma_E$ relativa all’intero modello.

\section{Modello SURE}

Nel modello SURE (\textit{Seemengly Uncorrelated Regression Equation}) si modifica il modello lineare generalizzato rendendolo più realistico: degli $r$ regressori si usano solo quelli effettivamente legati alle diverse variabili dipendenti, rendendo diversi i regressori tra le varie equazioni.
Inoltre questo modello potrebbe facilitare la gestione dei valori mancanti eliminando regressori poco significativi per alcune variabili.
Quindi la somma di tutti i regressori nelle diverse equazioni è uguale a
\begin{equation*}
\sum_{j=1}^n r_j
\end{equation*}
Data $n_j$ come la numerosità delle osservazioni per l’equazione $j$-esima allora il complesso delle numerosità è dato da 
\begin{equation*}
n = \sum_{j=1}^m n_j
\end{equation*}
La soluzione dei minimi quadrati per la stima dei coefficienti sembra simile a quella dei minimi quadrati generalizzati ma solo in apparenza perché il modello è caratterizzato dalla presenza di variabili esplicative diverse da equazione ed equazione.
Gli errori sono ipotizzati essere:
\begin{itemize}[noitemsep]
\item omoschedastici e incorrelati nella stessa equazione;
\item eteroschedastici fra diverse equazioni;
\item correlati per lo stesso individuo e incorrelati tra individui diversi fra diverse equazioni.
\end{itemize}
Rispetto al metodo OLS, lo stimatore dei parametri SURE presenta valori nulli, per eliminare i regressori poco significativi.
La rimozione di variabili esplicative influisce anche sugli altri parametri del modello, quindi $b_{i.j} \ne b_{k.j} \hspace{10pt} i \ne k$.

\section{Il problema dei dati gerarchici e uso di Regressione multilevel}
I modelli statistici, di solito, si basano sull'assunzione di indipendenza delle osservazioni, ottenuta per mezzo di un \textit{campionamento casuale}.
In questo caso si dice che le osservazioni si distribuiscono come una variabile casuale e che sono \textit{iid}, ovvero \textit{identicamente ed indipendentemente distribuite}.

In molti casi, però i dati risultano essere raggruppati in cluster naturali ovvero si distribuiscono naturalmente in una struttura gerarchica come ad esempio:
\begin{itemize}[noitemsep]
\item  i pazienti negli ospedali;
\item  gli studenti nelle classi;
\item gli impiegati nelle aziende.
\end{itemize}
In tali casi il campionamento casuale semplice non risulta efficiente, ma appare preferibile effettuare un \textit{campionamento a più stadi} perché si desidera analizzare le relazioni tra le variabili che possono essere misurate a livelli di raggruppamento diversi (livelli gerarchici della struttura dei dati).
Questo tipo di campionamento implica infatti una dipendenza tra le osservazioni appartenenti allo stesso gruppo (ad esempio gli studenti appartenenti alla stessa scuola condividono lo stesso ambiente, gli stessi insegnanti, lo stesso quartiere di provenienza e interagiscono tra di loro).

Quando i dati possiedono una struttura gerarchica significa che possono essere scomposti in dati \textit{dell'unità} e dati \textit{del gruppo}.
La dipendenza tra le unità di primo livello (micro) appartenenti alla stessa unità di secondo livello (macro) è cruciale per l’analisi.
\newline
Ignorando la struttura naturale dei dati si cadrebbe nel paradosso di Simpson, secondo cui i valori osservati su una variabile dipendono dal raggruppamento effettuato sui dati.
Infatti ignorando la struttura naturale dei dati si può perdere informazione in due modi:
\begin{enumerate}[noitemsep]
\item aggregando i dati micro (le unità) a livello macro (le sovrastrutture), andando incontro a quella che è definita \textit{fallacia ecologica}: se vi è correlazione tra variabili macro non può essere usata per fare asserzioni a livello micro.
\item disaggregando i dati ignorando la variabilità tra i gruppi, andando incontro a quella che è definita \textit{fallacia atomistica}: se vi è correlazione tra variabili a livello \textit{micro} non può essere usata per fare asserzioni a livello \textit{macro}.
\end{enumerate}

Ipotizzando una \textit{regressione Multilevel} ed ipotizzando che i dati abbiano struttura ad un livello, nulla cambia rispetto al Modello lineare classico univariato, singolare o multiplo, infatti la formula multilevel assume la forma:
\begin{equation*}
y = X \beta + r
\end{equation*}

Con un raggruppamento, si possono effettuare regressioni a livello macro sulla media delle osservazioni:
\begin{equation*}
  \bar{y} = \bar{x} \beta + r
\end{equation*}
dove ogni osservazione $j$ del vettore dei dati corrisponde alla media del $j$-esimo gruppo.

Si possono inoltre effettuare delle regressioni a livello micro, per ogni gruppo $j$:
\begin{equation*}
  y_j - \bar{y}_j = X_j \alpha + r_j
\end{equation*}
Con $\tilde{X}_j$ matrice dei dati per il $j$-esimo gruppo centrato sulla sua media $\bar{x}_j$ e $\alpha$ costante per tutti i gruppi: si tenta di identificare dei coefficienti comuni indipendenti dal partizionamento.
Si può effettuare la stessa analisi con una mera regressione che permetta ai coefficienti $\beta$ di variare tra i gruppi:
\begin{equation*}
  y_j = X_j \beta_j + r_j
\end{equation*}
Se il vettore $\beta_j$ ha valori costanti per ogni gruppo $j$, allora la struttura gerarchica dei dati non è significativa e non c'è differenza con una semplice regressione OLS.
Al contrario, se il partizionamento è significativo, al variare dell'intercetta si ha un modello \textit{random intercept}, mentre al variare degli altri coefficienti si ha un modello \textit{random coefficient}.

Inoltre si può effettuare una regressione \textit{multilevel}:
\begin{equation*}
  y_j = \bar{x}_j \beta_j + X_j \alpha + r_j
\end{equation*}
con $\alpha$ costante.
Questa formula, che è la somma delle due precedenti, tiene conto sia della varianza nei gruppi che della varianza fra i gruppi.


\section{Modello Multilevel: definizione e significato}
Per effettuare l'analisi della \textit{covarianza} (ANCOVA) occorre prima di tutto partire dall'analisi della varianza (ANOVA):
\begin{equation*}
y_{j} = \gamma + u_j + r \hspace{15pt} j = 1 \dots p
\end{equation*}
dove $\gamma$ rappresenta la media delle medie dei gruppi, $u_j = \overline{y}_j - \gamma$ la differenza tra la media del gruppo e la media complessiva $\gamma$ ($\gamma + u_j$ quindi rappresenta la media del $j$-esimo gruppo) e $r$ la componente erratica individuale.
Il modello ANOVA tenta di spiegare quanto la variabilità di $y_j$ è dovuta alle differenze delle medie fra i gruppi $u_j$.
Questo modello infatti è utilizzato nel caso in cui le covariate $x_j$ siano tutte qualitative.

Dati:
\begin{align*}
y &= [y_1,\hspace{5pt} \dots \hspace{5pt} ,y_p] \\
r &=[r_1, \hspace{5pt} \dots \hspace{5pt}, r_p] 
\end{align*}
dove $y_j$ e $r_j$ sono vettori di lunghezza pari a quella del proprio gruppo $n_j$, costruendo la matrice di \textit{presenza-assenza} $A$, composta solo da vettori $0$ e $1$ (della medesima lunghezza $n_j$)
\begin{equation*}
  A=
\begin{bmatrix}
  1_{n_1 \times 1} & \cdots & 0_{n_j \times 1} & \cdots & 0_{n_p \times 1} \\
  \vdots & \ddots & \vdots & \iddots & \vdots \\
  0_{n_1 \times 1} & \cdots & 1_{n_j \times 1} & \cdots & 0_{n_p \times 1} \\
  \vdots & \iddots & \vdots & \ddots & \vdots \\
  0_{n_1 \times 1} & \cdots  & 0_{n_j \times 1} & \cdots & 1_{n_p \times 1}
\end{bmatrix}
\end{equation*}
si calcola la devianza della variabile risposta per il $j$-esimo gruppo considerando solamente la variazione del livello:
\begin{align*}
  Dev(y_j) &= (y_j - \gamma)^\prime (y_j - \gamma) \\
           &= (u_j A_{j.j} + r_j)^\prime (u_j A_{j.j} + r_j) \\
           &= n_j \cdot u_j^2 + r_j^\prime r_j
\end{align*}

Siccome la devianza interna ai gruppi $r_j$, ipotizzando omoschedasticità, è pari a $n_j \cdot \sigma^2$ e la devianza tra i gruppi corrisponde alla devianza delle medie di gruppo $n_j \cdot u_j^2$, si può quindi calcolare la devianza totale
\begin{align*}
  SST &= SSE + SSR \\
      &= n_j \cdot u_j^2 + n_j \cdot \sigma^2 = n_j(u_j^2 + \sigma^2)
\end{align*}
Distinguendo quindi la varianza spiegata $u_j^2$ dalla residua $\sigma^2$ (con ipotesi di omoschedasticità), si può calcolare il coefficiente di correlazione intraclasse $\varrho$:
\begin{equation*}
\varrho = \frac{u^2}{u^2 + \sigma^2} \in [-1, +1]
\end{equation*}

Si riconduce il modello ANOVA al modello lineare generalizzato ponendo:
\begin{align*}
  \beta &= u \\
  X &= A \\
  y &= y - \gamma \\
  \varepsilon &= r
\end{align*}

Il modello ANOVA però presuppone che le variabili indipendenti non influenzino la risposta $y_j \indep X_j$.
Per ovviare a questo problema si usa il metodo di analisi della covarianza ANCOVA, che tenta di eliminare l'effetto della dipendenza per poi procedere con l'analisi della varianza vera e propria.
Si procede quindi con $4$ fasi di preparazione:
\begin{enumerate}[noitemsep]
\item si calcola la devianza totale di $Y$;
\item si stimano i coefficienti di regressione;
\item si calcola la devianza spiegata di $X$ ($SSE$);
\item si stima la devianza residua corretta di $y$ uguale a $\widehat{SSR}_{y}=SST_y - SSE_x$.
\end{enumerate}
Successivamente l’analisi della varianza ANOVA cattura la relazione aggregata fra i gruppi descrivendone la varianza.
Effettuata l'analisi della varianza su $\widehat{SSR}_{y}$, si calcola al devianza spiegata dal \textit{fattore sperimentale} corretta per la dipendenza tra la matrice del disegno $X_j$ e la variabile risposta $y$.
Si ha quindi:
\begin{align*}
  SST_y &= SSE_x + \widehat{SST}_y \\
        &= SSE_x + \widehat{SSE}_y + \widehat{SSR}_y
\end{align*}

Questo modello elimina gli aspetti individuali concentrando l'analisi sugli aspetti di gruppo e sulla varianza dei gruppi.

Il modello ANOVA è specificabile anche in una versione a \textit{effetti casuali} (e una \textit{a effetti misti}), per cui si hanno le stesse strutture ma con $u_j$ manifestazione della variabile casuale $\upsilon_j \sim N_{n_j}(0, u^2)$ e $E_j \sim N_{n_j}(0, \sigma^2)$.
In questo caso il Test $F$ serve a verificare ($H_0$) che le medie parziali ottenute dal campione possano essere ritenute tra di loro uguali, confrontando gli intervalli di confidenza.

Nel caso si consideri l'intera popolazione, o anche se $\upsilon_j$ ed $E_j$ abbiano distribuzione non normale, è meglio utilizzare l'analisi della varianza ad effetti fissi.
Il modello ANCOVA ad effetti casuali è definito \textit{modello Multilevel}: restando valide tutte le considerazioni fatte fin ora, la regressione lineare (OLS) cattura la relazione disaggregata tra i dati dei vari gruppi, così da eliminare l'effetto distorsivo e ricavare la varianza nei gruppi; mentre l'analisi della covarianza ANCOVA cattura la relazione aggregata fra i gruppi e descrive quindi la varianza fra i gruppi.

In un primo tipo di modelli (\textit{Mixed Models}) la relazione disaggregata tra i dati e la varianza nei gruppi sono descritte mediante parametri fissi mentre la relazione aggregata fra i gruppi e la varianza fra gruppi sono descritte come variabili casuali.

In un secondo tipo di modelli (\textit{Random Models}) anche la relazione disaggregata tra i dati e la varianza nei gruppi sono descritte come variabili casuali.

Tutti i modelli lineari possono essere riassunti come casi particolari del modello \textit{Multilevel}:
\begin{itemize}
\item Senza nessuna gerarchia nei dati si ha un modello lineare: 
\begin{equation*}
y = X \beta + \varepsilon
\end{equation*}
\item Con gerarchia nei dati si ha una regressione Multilevel: 
\begin{equation*}
y_j = X_j \beta_j + \varepsilon_j
\end{equation*}
\item Per $SSE = 0$ e $\beta_j$ stocastico si ha un'analisi della varianza ANOVA: 
\begin{equation*}
y_j = \overline{y} + \beta_j + \varepsilon_j
\end{equation*}
\item Per $SSE = 0$ e $\beta_j$ variabile casuale si ha un'ANOVA a effetti casuali: 
\begin{equation*}
y_j = \overline{y} + \beta_j + \varepsilon_j.
\end{equation*}
\end{itemize}


\section{Modello Multilevel: OLS, Empty, Mixed, Total Effects}
La stima del modello Multilevel si compone di 4 step:
\begin{enumerate}[noitemsep]
\item si stima innanzitutto il modello lineare, solitamente con il metodo di stima OLS;
\item si propone poi l'\textit{Empty model} (anche detto \textit{Unconditional means model UMM}) ovvero l’analisi della varianza a effetti casuali;
\item si propone il \textit{random intercepts model} (RIM), ovvero l’analisi della covarianza a effetti casuali per l’analisi della varianza.
\item Random slopes and intercepts model (UGM) analisi della covarianza a effetti casuali sia per il modello lineare che per l’analisi della varianza.
\end{enumerate}

\subsection*{1. Stima modello lineare con metodo di stima OLS}
Si consideri un modello Multilevel in cui compare solamente la parte stimata mediante metodo OLS:
\begin{equation*}
  y_j = X_j \beta + \varepsilon_j
\end{equation*}
con gli errori $\varepsilon_j \sim N(0, \sigma^2)$.
Si ha così un modello in cui è ignorata la struttura gerarchica dei dati.

Si possono proporre anche regressioni Multilevel introducendo variabili esplicative $Z$ misurate sui gruppi, invece che sugli individui, e quindi rappresentanti il livello 2.
Questo aspetto inoltre può essere esteso anche all'interazione \textit{cross-level}, ciò significa che nel modello si possono introdurre variabili prodotto originate dall'interazione tra variabili misurate sull'individuo e misurate sui gruppi cui gli individui appartengono.
Per risolvere la regressione multilevel si può scomporre il coefficiente di regressione in parte \textit{between} e parte \textit{within}: 
\begin{equation*}
y_j = X_j \beta_{within} +\overline{x}_j\beta_{between} +\varepsilon
\end{equation*}

si calcola il \textit{contextual effect} $\delta = \beta_{between} - \beta_{within}$ che rappresenta l'effetto della media dei gruppi.

\subsection*{2. Empty model}
Nel modello ANOVA ad effetti casuali detto anche \textit{Empty Model} si ha che
\begin{equation*}
y_j = v_j + r_j = (\gamma + u_j) + r_j
\end{equation*}
In questo caso la variabile dipendente $y_j$ dipende dalla media di gruppo $v_j \sim N(\gamma, \tau^2)$ e da una componente residuale $r_j \sim N(0, \sigma^2)$ con $v_j$ ed $r_j$ indipendenti e mutualmente incorrelati.
La variabilità all’interno di ogni gruppo è quindi dovuta solamente alla distribuzione casuale della variabile dipendente.

La variabilità totale di $y$ può essere scomposta nella fra i gruppi $\tau^2$ e varianza nei gruppi $\sigma^2$:
\begin{equation*}
Var(y) = Var(u)+Var(r_j) = \tau^2 + \sigma^2
\end{equation*}
Si può quindi definire il \textit{coefficiente di correlazione intraclasse}:
\begin{equation*}
\varrho = \frac{\tau^2}{\tau^2 + \sigma^2} \in [0, +1]
\end{equation*}
che misura la quota di varianza di $y$ spiegata dall’appartenenza ai gruppi dei singoli individui.
Se $\varrho = 0$, ovvero tutti gli $u_j$ sono nulli (o i $v_j$ sono uguali tra di loro e a $\gamma$), allora il raggruppamento non è significativo ed è inutile utilizzare modelli più complessi del lineare semplice.
Nel caso invece $\rho$ fosse positivo, è necessario considerare un modello di tipo gerarchico.
Per verificare questa condizione, non si confrontano i singoli valori $u_j$ ma i loro intervalli di confidenza: due valori si considerano statisticamente uguali se il loro intervallo di confidenza ha punti in comune.
Si può effettuare un test $F$ per verificare l'ipotesi complessiva $H_0 : u_j = 0, \hspace{5pt} j = 1 \dots m$:
\begin{equation*}
  \frac{^{SSF}/_{p - 1}}{^{SSE}/_{n - p}} \sim F_{p - 1, n - p}
\end{equation*}

\subsection*{3. Random intercept model}
Considerando anche la variazione della variabile risposta in relazione ai dati del gruppo si ottiene il modello \textit{random intercept}:
\begin{equation*}
y_j = \gamma + u_j + X_j \beta + \varepsilon_j
\end{equation*}
dove $u_j \sim N(0, \tau^2)$ rappresenta i residui di secondo livello (ovvero di gruppo) non spiegabili da $X_j$ e sono incorrelati ai residui di primo livello $u_j \indep \varepsilon_j$.
In questo caso la variabile dipendente $y$ dipende sia dai dati del $j$-esimo livello $X_j$, sia dall'effetto casuale dei residui di secondo livello $u_j$ sia da quelli di primo $\varepsilon_j$.

Si può calcolare ancora la correlazione intraclasse $\varrho$, che rappresenta la quota di varianza di $y$ spiegata dall'appartenenza ai singoli gruppi rispetto, senza contare la quota spiegata dai dati $X_j$.
Per questo motivo, il valore di $\varrho$ può essere minore rispetto a quello dell'\textit{empty model}.

Il modello prevede la stima di $\gamma$, $\beta$, $\sigma^2$ e $\tau^2$; resta inoltre valida l'interpretazione di $\beta$ come variazione di $y$ rispetto ad una variazione unitaria di $X$.

Non importa, ai fini statistici, se le variabili esplicative relative al secondo livello sono misurate sulle unità o sui gruppi; inoltre non è nemmeno necessario che tutti i gruppi abbiano la stessa numerosità dato l'\textit{effetto shrinkage}, che modifica i pesi dei gruppi.

\subsection*{4. Random slopes and intercepts model} 
La relazione tra variabile dipendente $y$ e le variabili esplicative $X_j$ può variare tra i gruppi in modi diversi: si può infatti avere un’eterogeneità dei coefficienti di regressione $\beta$ tra i diversi gruppi (si parla anche di interazione gruppo - covariate).
Si può quindi costruire un modello dove anche i coefficienti $\beta$ variano in base al gruppo:
\begin{equation*}
y_j = X_j \beta_j + r_j
\end{equation*}
In base ai valori assunti dai singoli $\beta_j$ si possono ottenere diverse tipologie di modelli:
\begin{itemize}[noitemsep]
  \item se il vettore $\beta_j$ rimane costante in ogni modello, la struttura gerarchica non è statisticamente significativa e si ha un modello OLS;
\item se l'intercetta cambia in ogni modello $j$, si ha un modello \textit{random intercept};
\item se anche i coefficienti cambiano in ogni modello $j$, si ha un modello \textit{random coefficient}.
\end{itemize}
Il vettore $\beta_j$ può essere scomposto in due vettori rappresentanti uno la parte costante e uno la variabile:
\begin{equation*}
  \beta_j = \gamma + u_j
\end{equation*}
dove  $\gamma$ rappresenta la parte fissa e $u_j$ gli effetti di gruppo.
Il modello dunque diventa:
\begin{align*}
  y_j &= X (\gamma + u_j) + r_j \\
      &= X_j \gamma + X_j u_j + r_j
\end{align*}

\section{Metodi di stima e verifica di ipotesi}
\subsection*{Specificazione del modello e stima dei parametri}
La specificazione del modello comporta la scelta del modello più soddisfacente. Nel caso di modelli lineari gerarchici tutto ciò implica:
\begin{itemize}
\item Scelta delle variabili esplicative $x_j$ e delle interazioni della parte fissa;
\item Scelta dei coefficienti casuali con le strutture di covarianza per la parte random del modello.
\end{itemize}
I parametri da stimare nel modello random intercept sono:
\begin{itemize}
\item Coefficienti di regressione $\gamma_{00}$ e $\beta$;
\item Componenti di varianza, $\sigma^2$ e $\tau^2$; 
\item Gli effetti casuali $U_{0j}$ non sono parametri ma variabili casuali latenti, ovvero non direttamente osservabili. 
\end{itemize}
I metodi comunemente utilizzati per la stima dei parametri sotto l’assunzione che i residui $U_{0j}$ e $R_{ij}$ siano distribuiti normalmente sono il metodo del \textit{maximum likelihood} (ML) ed il \textit{restricted maximum likelihood} (REML).

Il metodo REML massimizza la verosimiglianza (likelihood) dei residui osservati ottenendo le stime degli effetti fissi usando metodi «non likelihood-like» come \textit{ordinary least squares} (OLS) o \textit{generalized least squares} (GLS) e, successivamente usa queste per massimizzare la verosimiglianza dei residui (sottraendo gli effetti misti) per ottenere le stime dei parametri della varianza.
\subsection*{Verifica di ipotesi}
\subsubsection*{Test sui parametri fissi del modello}
Per testare i parametri fissi del modello si utilizza la seguente ipotesi nulla (ipotesi di significatività) su ciascun parametro 
\begin{equation*}
H_0: \gamma_h = 0
\end{equation*}
\newline
Questa ipotesi viene verificata con un Test t
\begin{equation*}
\mathrm{T}(\gamma_h) = \frac{\hat{\gamma}_h}{\mathrm{s.e.(\hat{\gamma}_h)}}
\end{equation*}
noto come WALD TEST. 

Sotto l’ipotesi nulla il test ha approssimativamente una distribuzione t con g.d.l. basati sulla struttura multilevel dell’analisi. 
\subsubsection*{Test su più parametri della parte fissa del modello e parte random}
Per testare più parametri (fissi e random) del modello invece viene utilizzato il deviance test. 

Dalla stima del modello lineare con il metodo ML si ottiene la verosimiglianza del modello, da cui: 
\begin{equation*}
\mathrm{DEVIANCE} = -2 \cdot ln(\mathrm{Likelihood}) 
\end{equation*}
misura della bontà di adattamento ai dati del modello.

Solitamente la deviance viene interpretata in termini differenziali, ovvero si calcola la differenza tra le deviance di modelli alternativi. 

Si tratta di confrontare i valori osservati della variabile dipendente con i valori teorici di due modelli: 
\begin{enumerate}[noitemsep]
\item l'uno con le variabili esplicative di interesse e l’altro senza alcuna variabile (\textbf{empty-model}); 
\item l'uno con le variabili esplicative di interesse e l’altro che contiene “tanti parametri quante sono le osservazioni” (\textbf{saturated model}).
\end{enumerate}
Il confronto si basa sulla funzione di log-verosimiglianza: perciò indicate rispettivamente con $D_0$, $D_{mod}$, $D_{sat}$ le devianze calcolate per il \textbf{modello vuoto} (empty-model), il \textbf{modello considerato} e il \textbf{modello saturo}, valori di $D_{mod}$ più prossimi a $0$ che non a $D_0$ faranno propendere per ritenere “buono” il modello considerato. 

Ognuna delle devianze ha distribuzione asintotica $\chi^2$ (con $j$ gradi di libertà pari al numero delle variabili esplicative). Le loro differenze avranno distribuzione $\chi^2$ (con k gradi di libertà pari alla differenza del numero di variabili esplicative).

Se l’obiettivo è quello di sottoporre a verifica l’ipotesi che riguarda la nullità congiunta di tutti i coefficienti (esclusa l’intercetta),
\begin{equation*}
H_0 : \beta_1 = \beta_2 = \dots = 0
\end{equation*}
si può pensare a ragion veduta di operare un confronto fra due modelli: l’empty-model e il modello ipotizzato. 
Questo test può essere applicato sia alla parte fissa sia a quella random del modello.
% \newpage
% \part*{Dimostrazioni}
% \section*{Non Efficienza con Errori Eteroschedastici}
% Siano $\varepsilon^*$ gli errori eteroschedastici,  $E[b^*]=E[((X'X)^{-1}X'y-b)((X'X^{-1})X'y-b)'] =$\newline  $E[((X'X)^{-1}X'(X\bcancel{b}+\varepsilon^*)-\bcancel{b})((X'X^{-1})X'(X\bcancel{b}+\varepsilon^*)-\bcancel{b})'] = (X'X)^{-1}X'E[\varepsilon^* (\varepsilon^*)']((X'X)^{-1}X')' = \newline (X'X)^{-1}X' \Sigma_{\varepsilon^*}((X'X)^{-1}X')'$ e abbiamo che $\Sigma_{\varepsilon^*}$ varia al variare dell'indice. $ \hfill \square$

% \section*{Non collinarietà dopo la trasformazione ritardata}
% Il modello trasformato è 
% \[y^{\#}_t = \beta_{0} + \beta_1x_t^{\#} + w_t \] 
% quindi abbiamo che 
% \[ cov(w_t,w_{t-1}) = cov(\varepsilon_t^{\#} - \rho \varepsilon_{t-1}^\#, \varepsilon_{t-1}^\# - \rho \varepsilon_{t-2}^\#) =  \]
% \[ = cov(\varepsilon_{t}^\#,\varepsilon_{t-1}^\#) -\rho cov(\varepsilon_{t-1}^\#,\varepsilon_{t-1}^\#) - \rho cov(\varepsilon_{t}^\#,\varepsilon_{t-2}^\#) + \rho^2 cov(\varepsilon_{t-1}^\#,\varepsilon_{t-2}^\#) = \]
% \[ = \rho - \rho - \rho^3 + \rho^3  = 0 \]
% il risulta poteva anche essere intuitivo poiché ho eliminato la parte ritardata che era presente nella variabile e portava alla multicollinarietà. $\hfill \square$

\end{document}