\documentclass[11pt, twocolumn]{article}

\usepackage[italian]{babel}
\usepackage{amsmath}
\usepackage{amssymb}
\usepackage[margin=1in]{geometry}
\usepackage[utf8]{inputenc}
\counterwithin*{section}{part}

\title{\textbf{Statistical Modeling}}
\author{}
\date{}


\begin{document}
\maketitle
\begin{abstract}
  Lo scopo del corso è riuscire a muoversi con dimestichezza all'interno di un dataset: è privilegiata la teoria perché indipendente dalla piattaforma; inoltre sono presentati numerosi esercizi svolti e database di esempio.
  Durante il corso si generalizza il modello lineare classico andando oltre alle sue premesse, arrivando al modello lineare multi-livello (che costruisce una gerarchia nei dati).

  L'esame è composto da una parte teorica di due domande (da un database di 15 note) e un esercizio da eseguire in R o SAS.
  Le slide sono sufficienti alla preparazione dell'esame, ma in più è offerta una dispensa ufficiale.
\end{abstract}

\newpage
\part{Premesse all'analisi}
Il modello stabilisce cosa fare coi dati: la finalità del modello stabilisce l'interpretazione da dare al risultato e i dati da raccogliere.
I test sul modello devono essere fatti su un campione \textit{significativo}: i risultati potrebbero non essere veritieri.
Il campione è necessario anche coi \textit{Big Data}, dato che aumentano l'eterogeneità dei dati.

Nella prima parte della costruzione del modello, lo statistico deve collaborare con l'esperto di dominio per individuare il fine del modello e i caratteri da osservare.
In un secondo momento si procede con un'analisi descrittiva (o esplorativa) del dataset (tramite grafici come istogrammi o boxplot, oppure calcolando valori indice).
Si individuano dunque gli \textit{outliers} (ovvero i valori anomali), da eliminare prima dell'analisi vera e propria.

Si analizza poi la matrice di correlazione $R$, per eliminare casi di \textit{multicollinearità} (ovvero i casi in cui la correlazione $\varrho \to 1$).

Si deve sciogliere il conflitto tra adattamento dei dati e parsimonia in modo tale da rendere comprensibile il modello ma garantendo una certa qualità nell'adattamento.

Il modello reale tiene conto dunque dell'errore, che considera cosa non è conoscibile e di componenti sistematici (esclusione di variabili) o di errori di misurazione; dai campioni (poco significativi) inoltre si può avere un errore stocastico.


Il modello statistico ha come scopo la comprensione e la minimizzazione dell'errore.
Non si arriva a formulare regole di causa-effetto ma solamente a relazioni empiriche.

Si stabilisce il modello di regressione individuando le variabili (e il loro grado) e il valore dei parametri nel vettore $\textbf{b}$.
Dunque si calcolano i valori stimati sui valori noti ($\hat{y_i}$) e si calcola l'errore con la formula $\varepsilon_i = y_i - \hat{y_i}$.
Al variare del campione nella popolazione, i valori dei parametri cambiano: è così possibile costruire una distribuzione (col metodo Montecarlo) di tali parametri.
L'errore dunque non è deterministico ma stocastico (cioè casuale), dato che varia in base al campione selezionato.

Il criterio scelto per ricercare $\textbf{b}$ è il criterio dei minimi quadrati ordinari che rende minima la norma del vettore di scarti $\varepsilon$

\newpage
\part{Modello lineare classico}
Il modello lineare classico ha delle premesse molto rigide a causa della sua natura.
Si tratta perlopiù di \textit{ipotesi semplificatrici} che saranno rimosse con modelli più avanzati (tranne per le ultime due).
La formula è lineare:
\begin{equation*}
  y = Xb + \varepsilon
\end{equation*}

\subsection*{Linearità.}
Le variabili e i parametri del modello sono lineari.

\subsection*{Non sistematicità degli errori}
Il vettore casuale ha valore atteso nullo (altrimenti il nostro modello non è imparziale):
\begin{equation*}
  E(\varepsilon_i)=0, i = 1,\hdots,n 
\end{equation*}
% $E(\varepsilon|X)=0$
e quindi segue che:
\begin{align*}
  E(y | X) &= E(Xb + \varepsilon) = \\
           &= Xb + E(\varepsilon) = \\
           &= Xb + 0 = Xb
\end{align*}

\subsection*{Sfericità degli errori.}
Gli errori sono omoschedastici (cioè la varianza è costante) e non correlati:
\begin{align*}
  E(\varepsilon) &= E(y - Xb) \\
                 &= E(Xb - Xb) = 0
\end{align*}
e quindi:
\begin{align*}
  &var(\varepsilon_i) = E(\varepsilon_i^2) = \sigma_i^2 \\
  &cov(\varepsilon_i, \varepsilon_j) = E(\varepsilon_i\varepsilon_j) = 0)
\end{align*}
Di fatto molti casi reali sono correlati tra di loro (ovvero un'osservazione influenza le altre), come nella finanza, nelle serie temporali o nelle coordinate spaziali.

\subsection*{Non omoschedasticità.}
È una mera convenzione: si presume che la matrice del disegno $X$ sia fissa e nota di dimensione $n \times p+1$ ; questa astrazione garantisce una semplificazione del problema.
Le componenti non stocastiche sono riassunte dalla componente erratica.

\subsection*{Non stocasticità delle variabili esplicative}
I valori delle $x_j$ variabili esplicative non sono soggetti a fluattuazioni da campione a campione, perciò $E(X) = X$ e $Cov(X,\epsilon) = 0$. La parte non fissa delle $x_j$ finisce in $\epsilon$.

\subsection*{Normalità degli errori.}
Il soddisfacimento dell'ipotesi che $\epsilon_i \sim N(0,\sigma^2)$ ovvero che la distribuzione dell’errore campionario sia normale provoca anche la normalità nella distribuzione di $y$ e di $\hat{\beta}$.
Questo permette la formulazione di test e la costruzione di intervalli di confidenza.


\subsection*{Non collinearità.}
Le variabili della matrice $X$ sono linearmente indipendenti. $X$ ha rango uguale al numero delle colonne (i caratteri, più una costante). Da ciò deriva che $X'X$ non è singolare: in caso contrario $X'X$ non è invertibile e il modello non risolvibile.

\subsection*{Numerosità della popolazione.}
Il numero di osservazioni è sempre maggiore del numero di caratteri osservati: $n \geq p + 1$.
Se questa proprietà non è soddisfatta, la matrice del disegno non è invertibile e dunque non è possibile la costruzione di alcun modello.


\section{Stima dei parametri.}
Per stimare i parametri, la formula è:
\begin{equation*}
  b = HX = (X'X)^{-1}X
\end{equation*}

Si dice che uno stimatore è \textit{corretto} se il valore medio della sua distribuzione è pari a quello della popolazione ($E(\theta) = E(x)$).
Invece per \textit{consistenza} si intende che con una popolazione infinita il valore stimato sia quello della popolazione.
In sostanza:
$$\lim_{n\to\infty} Var(\hat{\theta}) = 0$$

Tra due (o più) stimatori è preferibile lo stimatore più \textit{efficiente}, ovvero con una varianza minore nella sua distribuzione.
Generalmente uno stimatore segue una distribuzione normale, supponiamo $\beta_j \sim N(\mu,\sigma^2)$ (ha una distribuzione normale), si effettua un test (con probabilità dell'errore di primo grado $\alpha$ arbitrario) per verificare la significatività di un parametro $\beta_j$ con varianza $\sigma$ nota:
\begin{align*}
  &P[-Z_{\frac{\alpha}{2}} < \frac{\beta_j}{\frac{\sigma}{\sqrt{n\sigma^{-1}_{j.j}}}} < +Z_{\frac{\alpha}{2}}] = \\
  &= P[-Z_{\frac{\alpha}{2}} \frac{\sigma}{\sqrt{n\sigma^{-1}_{j.j}}} < \beta_j < Z_{\frac{\alpha}{2}} \frac{\sigma}{\sqrt{n\sigma^{-1}_{j.j}}}] = 1 - \alpha
\end{align*}
Solitamente nel modello lineare l'ipotesi nulla ($H_0$) è l'ipotesi che il parametro sia nullo $\beta_j \sim N(0,\sigma^2)$.

I test statistici per la significatività sono generalmente $F$ e $T$ (il primo è il quadrato del secondo).
\end{document}