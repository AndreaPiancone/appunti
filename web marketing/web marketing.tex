\documentclass[11pt, twocolumn]{article}

\usepackage[italian]{babel}
\usepackage{amsmath}
\usepackage{amssymb}
\usepackage[margin=1in]{geometry}
\usepackage[utf8]{inputenc}
\counterwithin*{section}{part}

% poco chiaro, non è vero ma ci sono stati casini sulla scelta delle lezioni ->
% lezione anche di mercoledì (a partire dalla terza settimana)
%
% email: (@unimib.it o @jakala.com)
% nico.didomenica@unimib.it
% attilio.revivo@unimib.it
% tutor data science: giovanni.collini@unimib,it
% tutor per gli altri: marta.laudiano@unimib.it

\title{\textbf{Web Marketing}}
\author{}
\date{}


\begin{document}
\maketitle
\begin{abstract}
  Per i frequentanti l'esame consisterà in un progetto che farà media con lo scritto, con possibilità di orale integrativo.
  Il progetto (fatto in gruppi di tre persone) per Data Science sarà un progetto in R di elaborazione dei dati con relativa presentazione.
  Altrimenti l'esame consiste in una presentazione frontend e backend di un'app (pre-esistente) proposta da un cliente.

  Per i non frequentanti l'esame consiste in uno scritto (composto da 10 domande aperte) e un orale.
  L'esame comunque verte solamente sulle slide.
\end{abstract}

% @momolo {{{
% mi spiace ragazzi, vado a fare foundation of probability...
% se volete continuare la dispensa avete tutta la mia stima :)
% }}}

\end{document}